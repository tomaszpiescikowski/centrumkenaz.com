% ══════════════════════════════════════════════════════════════════════════════
%  INSTRUKCJA UŻYTKOWNIKA — Kenaz Centrum
%  Kompilacja: xelatex INSTRUKCJA_UZYTKOWNIKA.tex  (dwukrotnie)
% ══════════════════════════════════════════════════════════════════════════════
\documentclass[11pt, a4paper]{article}

% ── Fonty ─────────────────────────────────────────────────────────────────────
\usepackage{fontspec}
\setmainfont{Palatino}[Ligatures=TeX, Numbers=OldStyle]
\setsansfont{Helvetica Neue}[BoldFont={Helvetica Neue Bold}, Scale=MatchLowercase]
\setmonofont{Courier New}[Scale=0.88]

% ── Język ─────────────────────────────────────────────────────────────────────
\usepackage{polyglossia}
\setdefaultlanguage{polish}

% ── Marginesy ─────────────────────────────────────────────────────────────────
\usepackage[a4paper, top=3cm, bottom=3cm, left=2.8cm, right=2.8cm, headheight=15pt]{geometry}

% ── Kolory ────────────────────────────────────────────────────────────────────
\usepackage{xcolor}
\definecolor{kenaznavy}{HTML}{0F174A}
\definecolor{kenazred}{HTML}{E53935}
\definecolor{kenazdim}{HTML}{4B5385}
\definecolor{tableshade}{HTML}{F5F4EF}
\definecolor{tipbg}{HTML}{EEF0F8}
\definecolor{partbg}{HTML}{F5F4EF}

% ── Hiperłącza ────────────────────────────────────────────────────────────────
\usepackage[unicode, colorlinks=true, linkcolor=kenaznavy,
            urlcolor=kenazred, citecolor=kenazdim,
            pdftitle={Instrukcja Użytkownika — Kenaz Centrum},
            pdfauthor={Kenaz Centrum}]{hyperref}

% ── Interlinia / akapity ──────────────────────────────────────────────────────
\usepackage{setspace}
\setstretch{1.4}
\setlength{\parskip}{6pt plus 2pt minus 1pt}
\setlength{\parindent}{0pt}

% ── Łamanie stron ─────────────────────────────────────────────────────────────
\widowpenalty=10000
\clubpenalty=10000
\displaywidowpenalty=10000

% ── Nagłówki i stopki ─────────────────────────────────────────────────────────
\usepackage{fancyhdr}
\pagestyle{fancy}
\fancyhf{}
\fancyhead[L]{\small\color{kenazdim}\nouppercase{\leftmark}}
\fancyhead[R]{\small\color{kenazdim}Kenaz Centrum}
\fancyfoot[C]{\small\color{kenazdim}---\;\thepage\;---}
\renewcommand{\headrulewidth}{0.3pt}
\renewcommand{\footrulewidth}{0pt}
\renewcommand{\headrule}{{\color{kenazdim!40}\hrule width\headwidth height 0.3pt}}
\fancypagestyle{plain}{%
  \fancyhf{}%
  \fancyfoot[C]{\small\color{kenazdim}---\;\thepage\;---}%
  \renewcommand{\headrulewidth}{0pt}%
}

% ── Nagłówki sekcji ───────────────────────────────────────────────────────────
\usepackage{titlesec}
\usepackage{needspace}
\titleformat{\section}
  {\sffamily\LARGE\bfseries\color{kenaznavy}}{\thesection}{0.8em}{}
\titleformat{\subsection}
  {\sffamily\large\bfseries\color{kenaznavy}}{\thesubsection}{0.8em}{}
\titleformat{\subsubsection}
  {\sffamily\normalsize\bfseries\color{kenazdim}}{\thesubsubsection}{0.8em}{}
\titlespacing{\section}     {0pt}{24pt plus 6pt minus 4pt}{8pt}
\titlespacing{\subsection}  {0pt}{16pt plus 4pt minus 2pt}{5pt}
\titlespacing{\subsubsection}{0pt}{10pt plus 2pt minus 2pt}{3pt}
\preto\section{\needspace{5\baselineskip}}
\preto\subsection{\needspace{4\baselineskip}}

% ── Nagłówki części (CZĘŚĆ I, II, ...) ───────────────────────────────────────
\newcommand{\czesc}[1]{%
  \clearpage
  \thispagestyle{plain}
  \vspace*{1.2cm}%
  \noindent
  \begin{tcolorbox}[colback=kenaznavy, colframe=kenaznavy,
                    left=16pt, right=16pt, top=14pt, bottom=14pt,
                    sharp corners, boxrule=0pt]
    {\sffamily\Huge\bfseries\color{white}#1\par}
  \end{tcolorbox}
  \vspace{0.8cm}%
  \addcontentsline{toc}{part}{#1}%
  \markboth{#1}{}%
}

% ── Spis treści ───────────────────────────────────────────────────────────────
\usepackage{tocloft}
\renewcommand{\cfttoctitlefont}{\sffamily\Large\bfseries\color{kenaznavy}}
\renewcommand{\cftsecfont}{\sffamily\bfseries\color{kenaznavy}}
\renewcommand{\cftsecpagefont}{\sffamily\small\color{kenazdim}}
\renewcommand{\cftsubsecfont}{\sffamily\small}
\renewcommand{\cftsubsecpagefont}{\sffamily\small\color{kenazdim}}
\setlength{\cftbeforesecskip}{2pt}
% Wpisz "część" w TOC jako grubszy nagłówek
\renewcommand{\cftpartfont}{\sffamily\bfseries\color{kenaznavy!80}}
\renewcommand{\cftpartpagefont}{\sffamily\small\color{kenazdim}}
\setlength{\cftbeforepartskip}{8pt}

% ── Tabele ────────────────────────────────────────────────────────────────────
\usepackage{booktabs}
\usepackage{longtable}
\usepackage{array}
\usepackage{colortbl}
\usepackage{tabularx}
\newcolumntype{L}[1]{>{\raggedright\arraybackslash}p{#1}}
\newcolumntype{C}[1]{>{\centering\arraybackslash}p{#1}}
\arrayrulecolor{kenaznavy!20}
\renewcommand{\arraystretch}{1.3}

% ── Ramki informacyjne ────────────────────────────────────────────────────────
\usepackage{tcolorbox}
\tcbuselibrary{skins, breakable}
\newtcolorbox{tipbox}{
  breakable,
  colback=tipbg,
  colframe=kenaznavy!35,
  boxrule=0.4pt,
  left=8pt, right=8pt, top=5pt, bottom=5pt,
  fontupper=\small,
  before upper={\setlength{\parskip}{3pt}},
}
\newtcolorbox{warnbox}{
  breakable,
  colback=kenazred!5,
  colframe=kenazred!40,
  boxrule=0.4pt,
  left=8pt, right=8pt, top=5pt, bottom=5pt,
  fontupper=\small,
  before upper={\setlength{\parskip}{3pt}},
}

% ── Listy ─────────────────────────────────────────────────────────────────────
\usepackage{enumitem}
\setlist[itemize]  {leftmargin=1.5em, itemsep=2pt, topsep=3pt, parsep=0pt}
\setlist[enumerate]{leftmargin=1.5em, itemsep=2pt, topsep=3pt, parsep=0pt}

% ── Mikrotypografia ───────────────────────────────────────────────────────────
\usepackage{microtype}

% ══════════════════════════════════════════════════════════════════════════════
\begin{document}

% ── Strona tytułowa ───────────────────────────────────────────────────────────
\thispagestyle{empty}
\begin{center}
  \vspace*{2.5cm}
  {\sffamily\fontsize{40}{48}\selectfont\bfseries\color{kenaznavy} Kenaz\par}
  \vspace{0.3cm}
  {\color{kenaznavy!25}\rule{\linewidth}{1.5pt}\par}
  \vspace{0.6cm}
  {\sffamily\Huge\bfseries\color{kenaznavy} Instrukcja Obsługi\par}
  \vspace{0.4cm}
  {\sffamily\Large\color{kenazdim} Przewodnik Użytkownika\par}
  \vspace{1.2cm}
  {\color{kenaznavy!25}\rule{0.4\linewidth}{0.5pt}\par}
  \vspace{0.6cm}
  {\sffamily\normalsize\color{kenazdim}
    \textbf{Wersja dokumentu:} 2.0 \quad
    \textbf{Data:} luty 2026\par}
  \vspace{0.3cm}
  {\sffamily\small\color{kenazdim!80}
    Dotyczy: aplikacja webowa i PWA Kenaz Centrum\par}
  \vfill
  {\sffamily\small\color{kenazdim!60} Kenaz Centrum\par}
\end{center}
\clearpage

% ── Spis treści ───────────────────────────────────────────────────────────────
\tableofcontents
\clearpage

% ══════════════════════════════════════════════════════════════════════════════
\czesc{CZĘŚĆ I — PIERWSZE KROKI}

% ──────────────────────────────────────────────────────────────────────────────
\section{Czym jest Kenaz?}

\textbf{Kenaz Centrum} to platforma społecznościowa dla aktywnej społeczności
lokalnej. Aplikacja umożliwia:

\begin{itemize}
  \item przeglądanie i zapisywanie się na \textbf{wydarzenia} (treningi,
        warsztaty, wycieczki i inne aktywności),
  \item komunikowanie się z uczestnikami za pomocą \textbf{wbudowanego chatu},
  \item zarządzanie własną \textbf{subskrypcją} i historią rejestracji,
  \item wspieranie centrum poprzez \textbf{darowizny} na stronie
        „Wesprzyj nas".
\end{itemize}

\subsection{Jak używać tej instrukcji?}

Każdy rozdział opisuje \textbf{jedną stronę lub funkcję} aplikacji.
Przy każdym opisie znajdziesz:

\begin{itemize}
  \item \textbf{Adres URL} danej strony (np.\ \texttt{/calendar}),
  \item \textbf{Kto ma dostęp} do tej strony,
  \item \textbf{Opis funkcji} krok po kroku,
  \item \textbf{Wskazówki} i \textbf{ważne uwagi}.
\end{itemize}

% ──────────────────────────────────────────────────────────────────────────────
\section{Rejestracja i logowanie}

\textbf{Adres:} \texttt{/login} \quad
\textbf{Dostęp:} wszyscy (w tym niezalogowani)

\subsection{Jak zalogować się do Kenaz?}

Kenaz używa wyłącznie logowania przez \textbf{Google}. Aby skorzystać
z aplikacji:

\begin{enumerate}
  \item Otwórz stronę główną aplikacji lub przejdź bezpośrednio pod adres
        \texttt{/login}.
  \item Kliknij przycisk \textbf{„Zaloguj się przez Google"}.
  \item Zostaniesz przekierowany do strony logowania Google. Wybierz swoje
        konto lub zaloguj się danymi Google.
  \item Po pierwszym zalogowaniu Twoje konto zostanie automatycznie
        \textbf{zarejestrowane} w systemie Kenaz.
  \item Zostaniesz przekierowany z powrotem do aplikacji.
\end{enumerate}

\begin{warnbox}
  \textbf{Ważne:} Przy pierwszym logowaniu konto trafia do statusu
  \textbf{„Oczekujące na akceptację"}. Korzystanie z pełnych funkcji
  (kalendarza, chatu, rejestracji) będzie możliwe dopiero po zatwierdzeniu
  konta przez administratora. Więcej na ten temat w rozdziale 4.
\end{warnbox}

\subsection{Hasła i resetowanie}

Kenaz nie używa haseł — logowanie odbywa się wyłącznie przez Google. Jeśli
nie pamiętasz hasła do swojego konta Google, skorzystaj z opcji odzyskiwania
konta dostępnej bezpośrednio na stronie Google.

\subsection{Wylogowanie}

\begin{enumerate}
  \item Przejdź do strony \textbf{„Moje konto"} (ikona osoby w nawigacji
        lub adres \texttt{/me}).
  \item Przewiń stronę na dół.
  \item Kliknij przycisk \textbf{„Wyloguj się"}.
\end{enumerate}

Po wylogowaniu zostaniesz przekierowany na stronę główną. Twoje dane
i rejestracje pozostaną zapisane w systemie.

% ──────────────────────────────────────────────────────────────────────────────
\section{Strona główna}

\textbf{Adres:} \texttt{/} \quad
\textbf{Dostęp:} wszyscy (w tym niezalogowani)

\subsection{Co zobaczysz na stronie głównej?}

Strona główna to \textbf{punkt startowy} aplikacji. Jej układ jest celowo
prosty — w centrum ekranu widnieje logo Kenaz Centrum, a poniżej trzy przyciski:

\begin{itemize}
  \item \textbf{„Zaloguj się"} — prowadzi do strony logowania (\texttt{/login}).
  \item \textbf{„O nas"} — prowadzi do strony z informacjami o organizacji
        (\texttt{/about}).
  \item \textbf{„Wesprzyj nas"} — prowadzi do strony darowizn
        (\texttt{/support}).
\end{itemize}

\subsection{Dla zalogowanych użytkowników}

Zalogowani użytkownicy zazwyczaj nie odwiedzają strony głównej — aplikacja
automatycznie kieruje ich do \textbf{kalendarza} (\texttt{/calendar})
po zalogowaniu. Strona główna służy głównie jako landing page dla nowych
odwiedzających.

\subsection{Tryb ciemny i jasny}

Strona główna obsługuje zarówno tryb jasny, jak i tryb ciemny. Możesz zmienić
motyw w ustawieniach konta lub klikając odpowiedni przełącznik w pasku
nawigacyjnym. Szczegóły w rozdziale 20.

% ──────────────────────────────────────────────────────────────────────────────
\section{Oczekiwanie na akceptację konta}

\textbf{Adres:} \texttt{/pending-approval} \quad
\textbf{Dostęp:} zalogowani użytkownicy ze statusem „oczekujące"

\subsection{Dlaczego moje konto oczekuje na akceptację?}

Kenaz to \textbf{zamknięta społeczność} — każde nowe konto musi zostać
zatwierdzone przez administratora. Dzięki temu platforma zachowuje wysoki
poziom bezpieczeństwa i spójności grupy.

Po pierwszym zalogowaniu przez Google Twoje konto automatycznie trafia
do kolejki oczekujących. Administrator widzi Twoje imię, adres e-mail
oraz datę rejestracji.

\subsection{Co widzę w czasie oczekiwania?}

\begin{itemize}
  \item \textbf{Kalendarz} jest widoczny, ale \textbf{zamazany} — nie możesz
        przeglądać szczegółów wydarzeń.
  \item \textbf{Panel} (lista Twoich rejestracji) jest widoczny,
        ale niedostępny.
  \item \textbf{Chat} jest niedostępny.
  \item Możesz przeglądać strony ogólne: \textbf{O nas}, \textbf{Wesprzyj
        nas}, \textbf{Politykę prywatności} i \textbf{Regulamin}.
\end{itemize}

\subsection{Jak długo trwa akceptacja?}

Czas akceptacji zależy od dostępności administratora. Zwykle odbywa się to
w ciągu \textbf{1–2 dni roboczych}. Jeśli po dłuższym czasie konto nadal nie
zostało zatwierdzone, skontaktuj się z organizacją.

\subsection{Co się dzieje po akceptacji?}

\begin{enumerate}
  \item Przy kolejnym zalogowaniu zobaczysz \textbf{ekran powitalny}
        z możliwością wyboru planu subskrypcji.
  \item Zostaniesz przekierowany do strony \textbf{Plany i subskrypcje}
        (\texttt{/plans?welcome=1}).
  \item Możesz od razu wybrać plan lub wrócić do konta i zdecydować później.
  \item Od tej chwili masz pełny dostęp do wszystkich funkcji aplikacji.
\end{enumerate}

\subsection{Blokada konta}

Konto może zostać \textbf{zablokowane} przez administratora. W takim
przypadku zostanie ono cofnięte do statusu „oczekującego" i wymagana
będzie ponowna akceptacja. Zablokowany użytkownik nie ma dostępu do płatnych
funkcji aplikacji.

% ──────────────────────────────────────────────────────────────────────────────
\section{Ustawienia i język interfejsu}

\subsection{Dostępne języki}

\begin{longtable}{@{}C{2cm}L{13cm}@{}}
\toprule
\rowcolor{kenaznavy!8}
\textbf{Kod} & \textbf{Język} \\
\midrule
\endhead
\bottomrule
\endfoot
\texttt{pl} & Polski (domyślny) \\
\rowcolor{tableshade}
\texttt{en} & English \\
\texttt{zh} & Chiński \\
\rowcolor{tableshade}
\texttt{nl} & Nederlands (niderlandzki) \\
\texttt{it} & Italiano (włoski) \\
\rowcolor{tableshade}
\texttt{sl} & Ślōnski (śląski) \\
\end{longtable}

\subsection{Jak zmienić język?}

Język możesz zmienić na stronie \textbf{Moje konto} (\texttt{/me}):

\begin{enumerate}
  \item Przejdź do \texttt{/me}.
  \item W sekcji ustawień znajdź \textbf{selektor języka}.
  \item Wybierz preferowany język z listy.
\end{enumerate}

Zmiana języka jest natychmiastowa i obowiązuje przez całą sesję.

\subsection{Jak zmienić miasto?}

Na stronie konta możesz wybrać swoje \textbf{miasto}. Ustawienie to wpływa
na wyświetlane wydarzenia — filtrując je do relewantnej lokalizacji.

\begin{enumerate}
  \item Przejdź do \texttt{/me}.
  \item W sekcji ustawień znajdź \textbf{selektor miasta}.
  \item Wybierz swoje miasto z listy.
\end{enumerate}

% ══════════════════════════════════════════════════════════════════════════════
\czesc{CZĘŚĆ II — GŁÓWNE FUNKCJE APLIKACJI}

% ──────────────────────────────────────────────────────────────────────────────
\section{Kalendarz wydarzeń}

\textbf{Adres:} \texttt{/calendar} \quad
\textbf{Dostęp:} wszyscy (pełny dostęp tylko zalogowani i zatwierdzeni)

\subsection{Ogólne informacje}

Kalendarz jest \textbf{głównym centrum aplikacji}. Pokazuje wszystkie nadchodzące
wydarzenia organizowane przez Kenaz — treningi, warsztaty, wykłady, wyjścia
i wiele innych aktywności.

Dla użytkowników niezalogowanych lub oczekujących na akceptację kalendarz jest
\textbf{widoczny, ale zamazany} — można zobaczyć ogólny zarys wydarzeń, ale
nie można wchodzić w szczegóły ani rejestrować się.

\subsection{Jak wygląda kalendarz?}

Kalendarz wyświetla wydarzenia w widoku \textbf{miesięcznym lub tygodniowym}.
Każdy dzień z wydarzeniami jest oznaczony. Możesz:

\begin{itemize}
  \item \textbf{Przeglądać dni} — klikając w datę zobaczysz listę wydarzeń.
  \item \textbf{Nawigować między miesiącami} — strzałkami góra/dół lub
        lewo/prawo.
  \item \textbf{Klikać w wydarzenie} — aby przejść do szczegółów.
\end{itemize}

\subsection{Oznaczenia wydarzeń na kalendarzu}

Każde wydarzenie może być oznaczone jedną z \textbf{ikon kategorii}. Ikony
pojawiają się obok nazwy wydarzenia i informują o jego typie (np.\ karate,
morsowanie, yoga).

\subsection{Filtrowanie według miasta}

Jeśli masz ustawione miasto w profilu, kalendarz może \textbf{automatycznie
filtrować} wydarzenia do Twojej lokalizacji. Możesz też zmienić filtr
bezpośrednio na stronie kalendarza.

\subsection{Prywatność i blokada}

Jeśli jesteś \textbf{niezalogowany} lub Twoje konto \textbf{oczekuje
na akceptację}: kalendarz jest zamazany (blur), kliknięcie w wydarzenie
nic nie robi, wyświetla się zachęta do zalogowania. Po zatwierdzeniu konta
uzyskujesz pełny dostęp.

% ──────────────────────────────────────────────────────────────────────────────
\section{Szczegóły wydarzenia}

\textbf{Adres:} \texttt{/event/:id} \quad
\textbf{Dostęp:} zalogowani i zatwierdzeni użytkownicy

\subsection{Jak otworzyć szczegóły wydarzenia?}

\begin{enumerate}
  \item Przejdź do kalendarza (\texttt{/calendar}).
  \item Kliknij w datę, aby zobaczyć listę wydarzeń danego dnia.
  \item Kliknij w nazwę wybranego wydarzenia.
  \item Zostaniesz przekierowany na stronę szczegółów: \texttt{/event/:id}.
\end{enumerate}

\subsection{Co zawiera strona szczegółów?}

\begin{longtable}{@{}L{4.5cm}L{10.5cm}@{}}
\toprule
\rowcolor{kenaznavy!8}
\textbf{Sekcja} & \textbf{Opis} \\
\midrule
\endhead
\bottomrule
\endfoot
\textbf{Tytuł i ikona}      & Nazwa wydarzenia z ikoną kategorii \\
\rowcolor{tableshade}
\textbf{Data i godzina}     & Dokładna data, godzina rozpoczęcia i zakończenia \\
\textbf{Miasto i lokalizacja} & Miejsce spotkania; przycisk „Otwórz w Google Maps" \\
\rowcolor{tableshade}
\textbf{Opis}               & Szczegółowy opis aktywności \\
\textbf{Cena}               & Cena dla gości lub informacja o bezpłatności \\
\rowcolor{tableshade}
\textbf{Dostępność miejsc}  & Liczba wolnych miejsc lub informacja o zapełnieniu \\
\textbf{Uczestnicy}         & Lista potwierdzonych uczestników \\
\rowcolor{tableshade}
\textbf{Lista oczekujących} & Osoby na liście oczekujących \\
\textbf{Przycisk rejestracji} & Zapis, anulowanie zapisu lub informacja o statusie \\
\rowcolor{tableshade}
\textbf{Komentarze}         & Sekcja komentarzy i dyskusji pod wydarzeniem \\
\textbf{Google Calendar}    & Przycisk dodania do własnego Google Calendar \\
\end{longtable}

\subsection{Dwie ceny — dla gości i dla członków}

Przy wydarzeniach płatnych możesz zobaczyć \textbf{dwie ceny}:

\begin{itemize}
  \item \textbf{Cena dla gości} — standardowa cena dla wszystkich uczestników
        bez subskrypcji.
  \item \textbf{Cena dla członków} — obniżona cena dla osób z aktywną
        subskrypcją Kenaz.
\end{itemize}

\begin{tipbox}
  \textbf{Wskazówka:} Jeśli regularnie uczestniczysz w wydarzeniach, opłaca
  się rozważyć wykupienie subskrypcji — ceny dla członków są zazwyczaj
  znacznie niższe lub zerowe.
\end{tipbox}

\subsection{Dodanie do Google Calendar}

Na stronie szczegółów wydarzenia znajdziesz przycisk \textbf{„Dodaj do Google
Calendar"}. Kliknięcie go otworzy nową kartę z formularzem Google Calendar,
gdzie możesz zapisać wydarzenie z automatycznie uzupełnioną datą, godziną
i lokalizacją.

\subsection{Komentarze pod wydarzeniem}

Każde wydarzenie posiada \textbf{sekcję komentarzy}, gdzie uczestnicy mogą
zadawać pytania i rozmawiać. Szczegółowy opis systemu komentarzy znajdziesz
w rozdziale 11.

% ──────────────────────────────────────────────────────────────────────────────
\section{Rejestracja na wydarzenie}

\textbf{Adres:} \texttt{/event/:id} \quad
\textbf{Dostęp:} zalogowani i zatwierdzeni użytkownicy

\subsection{Jak zapisać się na wydarzenie?}

\begin{enumerate}
  \item Przejdź na stronę szczegółów wybranego wydarzenia.
  \item Sprawdź dostępność miejsc.
  \item Kliknij przycisk \textbf{„Zapisz się"}.
  \item Jeśli wydarzenie jest \textbf{bezpłatne}: rejestracja zostanie
        potwierdzona natychmiast.
  \item Jeśli wydarzenie jest \textbf{płatne}: zostaniesz przekierowany
        do procesu płatności (patrz rozdział 9).
\end{enumerate}

\subsection{Stany rejestracji}

\begin{longtable}{@{}L{4.5cm}L{10.5cm}@{}}
\toprule
\rowcolor{kenaznavy!8}
\textbf{Status} & \textbf{Znaczenie} \\
\midrule
\endhead
\bottomrule
\endfoot
\textbf{Potwierdzony}          & Jesteś zapisany — masz zagwarantowane miejsce \\
\rowcolor{tableshade}
\textbf{Oczekuje na płatność}  & Zarejestrowany, ale płatność jeszcze nie potwierdzona \\
\textbf{Lista oczekujących}    & Brak miejsc — zostaniesz przeniesiony automatycznie \\
\rowcolor{tableshade}
\textbf{Anulowany}             & Rejestracja anulowana przez Ciebie lub administratora \\
\end{longtable}

\subsection{Jak anulować rejestrację?}

\textbf{Ze strony szczegółów wydarzenia:}

\begin{enumerate}
  \item Przejdź na \texttt{/event/:id}.
  \item Kliknij przycisk \textbf{„Anuluj rejestrację"}.
  \item Potwierdź anulowanie w okienku dialogowym.
\end{enumerate}

\textbf{Z panelu użytkownika:}

\begin{enumerate}
  \item Przejdź do \texttt{/panel}.
  \item Znajdź wydarzenie na liście swoich rejestracji.
  \item Kliknij przycisk anulowania i potwierdź.
\end{enumerate}

\begin{tipbox}
  \textbf{Uwaga przy anulowaniu płatnych rejestracji:} Jeśli zapłaciłeś
  przelewem, po anulowaniu zostaniesz poinformowany o warunkach ewentualnego
  zwrotu. Zwroty są procesowane przez administratora.
\end{tipbox}

\subsection{Limit uczestników i automatyczna kolejka}

Gdy limit zostanie osiągnięty, przycisk rejestracji zmieni się na
\textbf{„Dołącz do listy oczekujących"}. System \textbf{automatycznie}
awansuje Cię, gdy ktoś anuluje rejestrację.

% ──────────────────────────────────────────────────────────────────────────────
\section{Płatność za wydarzenie}

\textbf{Adresy:} \texttt{/event/:id}, \texttt{/manual-payment/:registrationId}

\subsection{Metody płatności}

Kenaz obsługuje \textbf{przelew bankowy (płatność manualna)} — większość
wydarzeń wymaga dokonania przelewu na podany numer konta i potwierdzenia
go w aplikacji.

\subsection{Przebieg płatności manualnej}

\begin{enumerate}
  \item Po kliknięciu \textbf{„Zapisz się"} system tworzy rejestrację
        ze statusem \textbf{„Oczekuje na potwierdzenie płatności"}.
  \item Aplikacja pokazuje link do strony płatności
        (\texttt{/manual-payment/:registrationId}).
  \item Na stronie płatności zobaczysz: \textbf{numer konta} bankowego,
        \textbf{tytuł przelewu}, \textbf{kwotę do zapłaty} i
        \textbf{termin płatności}.
  \item Wykonaj przelew w swoim banku.
  \item Wróć do aplikacji i kliknij \textbf{„Potwierdzam wykonanie przelewu"}.
  \item Administrator weryfikuje wpłatę i akceptuje rejestrację.
\end{enumerate}

\begin{warnbox}
  \textbf{Ważne:} Termin płatności jest automatycznie wyliczany przez system.
  Jeśli nie dokonasz przelewu i nie potwierdzisz go w wyznaczonym czasie,
  rejestracja może zostać automatycznie anulowana.
\end{warnbox}

\subsection{Strona potwierdzenia płatności}

Pod adresem \texttt{/manual-payment/:registrationId} znajdziesz stronę
ze szczegółami płatności: informację o wydarzeniu, dane do przelewu oraz
przycisk \textbf{„Potwierdzam wykonanie przelewu"}.

\subsection{Subskrypcja a ceny}

Jeśli posiadasz aktywną \textbf{subskrypcję Kenaz}, automatycznie korzystasz
z niższych cen lub możesz brać udział w bezpłatnych wydarzeniach dla
subskrybentów. Więcej informacji w rozdziale 14.

% ──────────────────────────────────────────────────────────────────────────────
\section{Lista oczekujących (waitlist)}

\subsection{Jak działa lista oczekujących?}

Gdy liczba zapisów na wydarzenie osiągnie \textbf{limit miejsc}, kolejne
osoby automatycznie trafiają na \textbf{listę oczekujących}. Lista jest
prowadzona w kolejności rejestracji (FIFO — kto pierwszy, ten lepszy).

\subsection{Automatyczne awansowanie}

System automatycznie awansuje pierwszą osobę z listy oczekujących, gdy
ktoś \textbf{anuluje} swoją rejestrację lub administrator ręcznie bierze
czyjejś miejsce. Awansowanie jest natychmiastowe — status Twojej rejestracji
zmienia się na \textbf{„Potwierdzony"} (bezpłatne) lub \textbf{„Oczekuje
na płatność"} (płatne).

\subsection{Jak sprawdzić swoje miejsce na liście?}

W panelu użytkownika (\texttt{/panel}) widzisz listę swoich rejestracji
wraz z aktualnym statusem przy każdej z nich.

% ──────────────────────────────────────────────────────────────────────────────
\section{Chat i komentarze}

\textbf{Adres:} \texttt{/chat} (główny chat), \texttt{/event/:id} (komentarze)\\
\textbf{Dostęp:} zalogowani i zatwierdzeni użytkownicy

\subsection{Dwa rodzaje komunikacji}

\begin{enumerate}
  \item \textbf{Chat ogólny i chat przy wydarzeniach} — dostępny pod adresem
        \texttt{/chat}.
  \item \textbf{Komentarze pod wydarzeniami} — wbudowane na stronie
        \texttt{/event/:id}.
\end{enumerate}

Oba systemy są oparte na tym samym mechanizmie. Chat (\texttt{/chat})
agreguje wszystkie wątki w jednym miejscu.

\subsection{Strona chatu (\texttt{/chat})}

\subsubsection{Widok: Lista wydarzeń z chatem}

Po wejściu na \texttt{/chat} zobaczysz listę \textbf{wydarzeń, na które
jesteś zapisany}. Przy każdym wydarzeniu widoczna jest liczba
\textbf{nieprzeczytanych wiadomości}. Możesz też przejść do
\textbf{czatu ogólnego} (Kenaz General) — globalnego kanału dla wszystkich.

\subsubsection{Widok: Rozmowa}

Po wybraniu wydarzenia lub kanału ogólnego przechodzisz do widoku rozmowy:

\begin{itemize}
  \item Lista wiadomości od najstarszej do najnowszej.
  \item Pole do wpisywania nowej wiadomości na dole ekranu.
  \item Przyciski dodawania \textbf{reakcji emoji} do wiadomości.
  \item Możliwość \textbf{odpowiadania} na konkretną wiadomość.
\end{itemize}

\subsection{Wysyłanie wiadomości}

\begin{enumerate}
  \item Przejdź do \texttt{/chat} i wybierz wydarzenie lub kanał.
  \item Wpisz wiadomość w polu na dole ekranu.
  \item Naciśnij \textbf{Enter} lub przycisk wysłania.
\end{enumerate}

Maksymalna długość wiadomości: \textbf{1000 znaków}. Nie można wysyłać
pustych wiadomości ani samych spacji.

\subsection{Odpowiadanie na wiadomości}

\begin{itemize}
  \item Na \textbf{desktop}: najedź kursorem na wiadomość $\rightarrow$
        pojawi się opcja „Odpowiedz".
  \item Na \textbf{mobile}: przytrzymaj wiadomość $\rightarrow$ pojawi się
        menu z opcją „Odpowiedz".
\end{itemize}

\subsection{Reakcje emoji}

\begin{itemize}
  \item Na \textbf{desktop}: najedź na wiadomość $\rightarrow$ kliknij
        ikonę emoji.
  \item Na \textbf{mobile}: przytrzymaj wiadomość $\rightarrow$ wybierz
        emoji z paska.
\end{itemize}

Kliknięcie ponownie tej samej reakcji \textbf{usuwa} Twoją reakcję.

\subsection{Edycja i usuwanie wiadomości}

Możesz \textbf{edytować} lub \textbf{usuwać} własne wiadomości (najedź
na wiadomość lub przytrzymaj ją na mobile). Edycja używa mechanizmu
wersjonowania — jeśli ktoś edytuje tę samą wiadomość równocześnie z Tobą,
starszy zapis zostanie odrzucony.

\subsection{Komentarze pod wydarzeniem}

Sekcja komentarzy dostępna jest na dole strony każdego wydarzenia
(\texttt{/event/:id}). Działają identycznie jak chat, ale są
\textbf{przypisane do konkretnego wydarzenia}.

\subsection{Nieprzeczytane wiadomości}

Ikona chatu w nawigacji pokazuje \textbf{czerwoną kropkę} lub
\textbf{liczbę} nieprzeczytanych wiadomości. Po wejściu w dany wątek
wiadomości są automatycznie oznaczane jako przeczytane.

% ══════════════════════════════════════════════════════════════════════════════
\czesc{CZĘŚĆ III — MOJE KONTO}

% ──────────────────────────────────────────────────────────────────────────────
\section{Panel — moje aktywności}

\textbf{Adres:} \texttt{/panel} \quad
\textbf{Dostęp:} zalogowani i zatwierdzeni użytkownicy

\subsection{Co to jest Panel?}

Panel to \textbf{centrum zarządzania Twoją aktywnością} w aplikacji.
Znajdziesz tu:

\begin{itemize}
  \item \textbf{Listę Twoich rejestracji} na nadchodzące i minione wydarzenia.
  \item \textbf{Statusy płatności} dla płatnych wydarzeń.
  \item \textbf{Linki do płatności manualnej} — jeśli oczekuje na Ciebie
        płatność.
  \item Opcję \textbf{anulowania rejestracji}.
\end{itemize}

\subsection{Lista rejestracji}

Rejestracje wyświetlane są chronologicznie (najbliższe na górze). Każda
karta rejestracji zawiera: \textbf{nazwę i datę} wydarzenia, \textbf{ikonę
kategorii}, \textbf{status rejestracji}, \textbf{cenę} i jej status
(opłacona / nieopłacona) oraz \textbf{przycisk akcji}.

\subsection{Widok dla niezalogowanych i niezatwierdzonych}

Strona panelu wyświetla \textbf{przykładowe, zamazane} karty wydarzeń jako
podgląd tego, jak będzie wyglądać panel po aktywacji konta.

\subsection{Anulowanie rejestracji z Panelu}

\begin{enumerate}
  \item Znajdź wydarzenie na liście.
  \item Kliknij przycisk \textbf{„Anuluj"} przy danej rejestracji.
  \item Pojawi się okno potwierdzenia — przeczytaj je uważnie.
  \item Potwierdź anulowanie.
\end{enumerate}

% ──────────────────────────────────────────────────────────────────────────────
\section{Profil użytkownika i ustawienia konta}

\textbf{Adres:} \texttt{/me} \quad \textbf{Dostęp:} zalogowani użytkownicy

\subsection{Przegląd strony konta}

Strona \texttt{/me} to Twój \textbf{profil osobisty} i centrum ustawień.
Podzielona jest na kilka sekcji.

\subsection{Sekcja: Dane podstawowe}

Znajdziesz tu: \textbf{Imię i nazwisko} (z konta Google, nieedytowalne),
\textbf{adres email}, \textbf{awatar}, \textbf{typ konta / rolę}
(Member, Admin lub Guest) oraz \textbf{status subskrypcji}.

\subsection{Sekcja: O mnie}

Możesz napisać \textbf{krótkie opisanie siebie}, które będzie widoczne
dla innych użytkowników na Twoim publicznym profilu.

Aby edytować: kliknij przycisk edycji $\rightarrow$ wpisz tekst $\rightarrow$
kliknij \textbf{„Zapisz"}.

\subsection{Sekcja: Zainteresowania}

Wybierz \textbf{tagi zainteresowań} — zaznaczone tagi są wyróżnione kolorem.
Kliknij ponownie, aby odznaczyć. Kliknij \textbf{„Zapisz zainteresowania"}.

\subsection{Sekcja: Ustawienia aplikacji}

Znajdziesz tu: \textbf{wybór języka}, \textbf{wybór miasta} oraz
\textbf{przełącznik trybu} jasny/ciemny.

\subsection{Sekcja: Subskrypcja}

Wyświetlana jest informacja o aktywnym planie (nazwa, data wygaśnięcia)
oraz przycisk \textbf{„Zarządzaj planem"} prowadzący do \texttt{/plans}.
Jeśli masz oczekującą płatność za subskrypcję, zobaczysz baner z linkiem
do strony płatności.

\subsection{Sekcja: Wylogowanie}

Na dole strony konta znajdziesz przycisk \textbf{„Wyloguj się"}.

% ──────────────────────────────────────────────────────────────────────────────
\section{Plany i subskrypcje}

\textbf{Adres:} \texttt{/plans} \quad
\textbf{Dostęp:} zalogowani i zatwierdzeni użytkownicy

\subsection{Co to jest subskrypcja?}

Subskrypcja Kenaz to \textbf{opcjonalny plan membership}, który daje dostęp
do preferencyjnych cen na wydarzeniach oraz innych benefitów.

\subsection{Dostępne plany}

\begin{longtable}{@{}L{3.5cm}L{7.5cm}L{4cm}@{}}
\toprule
\rowcolor{kenaznavy!8}
\textbf{Plan} & \textbf{Opis} & \textbf{Cena} \\
\midrule
\endhead
\bottomrule
\endfoot
\textbf{Darmowy}    & Podstawowy dostęp; standardowe ceny wydarzeń & 0 zł \\
\rowcolor{tableshade}
\textbf{Miesięczny} & Obniżone ceny; wygasa po miesiącu & np.\ 69 zł/mies. \\
\textbf{Roczny}     & Najlepsza wartość; najniższe ceny na cały rok & np.\ 599 zł/rok \\
\end{longtable}

\begin{tipbox}
  Dokładne ceny planów są konfigurowane przez administratorów i mogą się
  różnić od podanych przykładów.
\end{tipbox}

\subsection{Jak wybrać lub zmienić plan?}

\begin{enumerate}
  \item Przejdź do \texttt{/plans}.
  \item Kliknij kartę z interesującym Cię planem.
  \item Wybierz długość okresu (dla planów płatnych).
  \item Kliknij \textbf{„Wybierz plan"} / \textbf{„Przejdź do płatności"}.
  \item Zostaniesz przekierowany do strony płatności manualnej za subskrypcję.
\end{enumerate}

\subsection{Płatność za subskrypcję}

Adres: \texttt{/subscription-purchases/:purchaseId/manual-payment}

\begin{enumerate}
  \item Na stronie zobaczysz dane do przelewu: numer konta, kwotę i tytuł.
  \item Wykonaj przelew.
  \item Kliknij \textbf{„Potwierdzam wykonanie przelewu"}.
  \item Administrator weryfikuje wpłatę i aktywuje subskrypcję.
\end{enumerate}

\subsection{Rezygnacja z planu}

Przejdź do \texttt{/plans}, wybierz \textbf{„Plan darmowy"} i potwierdź.
Zmiana jest natychmiastowa.

\subsection{Modal powitalny (po pierwszym zatwierdzeniu konta)}

Gdy administrator zaakceptuje Twoje konto, zobaczysz \textbf{ekran powitalny}
z przyciskiem „Wybierz plan" i opcją odłożenia decyzji na później.

\subsection{Jak subskrypcja wpływa na ceny wydarzeń?}

Na stronie szczegółów każdego płatnego wydarzenia widoczne są dwie ceny:
\textbf{cena dla gości} (bez subskrypcji) i \textbf{cena dla członków}
(z subskrypcją). System automatycznie nalicza odpowiednią cenę.

% ──────────────────────────────────────────────────────────────────────────────
\section{Publiczny profil użytkownika}

\textbf{Adres:} \texttt{/people/:userId} \quad
\textbf{Dostęp:} zalogowani i zatwierdzeni użytkownicy

\subsection{Co to jest publiczny profil?}

Każdy użytkownik Kenaz ma \textbf{publiczny profil} obsahující: imię i awatar,
opis „O mnie" (jeśli uzupełniony) oraz tagi zainteresowań (jeśli wybrane).

\subsection{Jak wejść na profil innego użytkownika?}

Przez kliknięcie w imię autora wiadomości w chacie, lub bezpośrednio pod
adresem \texttt{/people/:userId}.

\subsection{Prywatność}

Na publicznym profilu \textbf{nie są wyświetlane} prywatne dane: adres email,
historia płatności, lista rejestracji. Widoczne są tylko te informacje,
które użytkownik sam udostępnił.

% ══════════════════════════════════════════════════════════════════════════════
\czesc{CZĘŚĆ IV — POZOSTAŁE STRONY}

% ──────────────────────────────────────────────────────────────────────────────
\section{Wesprzyj nas — darowizny}

\textbf{Adres:} \texttt{/support} \quad
\textbf{Dostęp:} wszyscy (zalogowani i niezalogowani)

\subsection{Po co ta strona?}

Strona \textbf{„Wesprzyj nas"} umożliwia wsparcie finansowe organizacji Kenaz.
Dostęp mają wszyscy — zarówno zalogowani, jak i anonimowi odwiedzający.

\subsection{Przelew bankowy}

Na stronie znajdziesz sekcję z danymi do przelewu:

\begin{itemize}
  \item \textbf{Numer konta bankowego} — kliknij ikonę kopiowania, aby
        skopiować numer jednym tapnięciem.
  \item \textbf{Tytuł przelewu} — wpisz go dokładnie, aby administrator
        mógł zidentyfikować wpłatę.
  \item \textbf{Dane odbiorcy} — pełna nazwa organizacji i adres.
\end{itemize}

\subsection{Zewnętrzne platformy wspierania}

Jeśli administrator skonfigurował zewnętrzne linki (np.\ buycoffee.to,
Patronite), zobaczysz przyciski prowadzące do tych platform.

\subsection{Formularz darowizny (dla zalogowanych użytkowników)}

\begin{enumerate}
  \item Wybierz kwotę z predefiniowanych opcji lub wpisz własną kwotę.
  \item Opcjonalnie dodaj krótką notatkę dla administratora.
  \item Kliknij \textbf{„Wesprzyj"}.
\end{enumerate}

\subsection{Punkty za darowiznę}

Każdy zalogowany użytkownik, który dokonuje darowizny przez formularz,
otrzymuje \textbf{punkty aktywności} dopisane do swojego profilu. Punkty
są przyznawane niezależnie od posiadanego planu.

\begin{tipbox}
  \textbf{Uwaga:} Darowizny manualne (przelew bankowy bez użycia formularza)
  wymagają ręcznej weryfikacji przez administratora przed przyznaniem punktów.
\end{tipbox}

% ──────────────────────────────────────────────────────────────────────────────
\section{O nas}

\textbf{Adres:} \texttt{/about} \quad \textbf{Dostęp:} wszyscy

\subsection{Co zawiera strona „O nas"?}

\begin{itemize}
  \item \textbf{Historię organizacji} — rozdziały ze zdjęciami i tekstami
        opisującymi początki i wartości Kenaz.
  \item \textbf{Statystyki} — kluczowe liczby (liczba członków, wydarzeń itp.).
  \item \textbf{Zdjęcia} — galeria z życia organizacji.
  \item \textbf{Przycisk dołączenia} — zachęta do zalogowania i dołączenia.
\end{itemize}

\subsection{Nawigacja po stronie}

Strona jest podzielona na sekcje w układzie tekstowo-fotograficznym.
Przewijaj stronę, aby zobaczyć wszystkie rozdziały historii.

\subsection{Karta „Wesprzyj nas"}

Na dole strony „O nas" wyświetlana jest karta z przyciskiem \textbf{„Wesprzyj
nas"}, który kieruje do strony darowizn (\texttt{/support}).

% ──────────────────────────────────────────────────────────────────────────────
\section{Polityka prywatności i Regulamin}

\textbf{Adresy:} \texttt{/privacy}, \texttt{/terms} \quad
\textbf{Dostęp:} wszyscy

\subsection{Polityka prywatności (\texttt{/privacy})}

Strona zawiera szczegółowe informacje dotyczące: jakie dane osobowe zbiera
aplikacja, w jakim celu są przetwarzane, jak długo są przechowywane, jakie
przysługują Ci prawa oraz informacje o cookies.

\subsection{Regulamin (\texttt{/terms})}

Strona zawiera warunki korzystania z aplikacji, m.in.: zasady uczestnictwa
w wydarzeniach, warunki subskrypcji i politykę zwrotów, zasady korzystania
z chatu oraz odpowiedzialność stron.

\begin{tipbox}
  Zalecamy zapoznanie się z obiema stronami przed pierwszym korzystaniem
  z aplikacji.
\end{tipbox}

% ══════════════════════════════════════════════════════════════════════════════
\czesc{CZĘŚĆ V — DODATKI}

% ──────────────────────────────────────────────────────────────────────────────
\section{Aplikacja mobilna (PWA)}

\subsection{Co to jest PWA?}

Kenaz jest dostępny jako \textbf{Progressive Web App (PWA)} — aplikacja webowa,
którą można zainstalować jak zwykłą aplikację. Nie wymaga pobierania ze sklepu
App Store ani Google Play.

\subsection{Jak zainstalować aplikację na telefonie?}

\textbf{Na Android (Chrome):}

\begin{enumerate}
  \item Otwórz aplikację Kenaz w przeglądarce Chrome.
  \item Kliknij baner \textbf{„Dodaj do ekranu głównego"}.
  \item Alternatywnie: trzy kropki w menu Chrome $\rightarrow$
        \textbf{„Dodaj do ekranu głównego"}.
  \item Potwierdź instalację.
\end{enumerate}

\textbf{Na iOS (Safari):}

\begin{enumerate}
  \item Otwórz aplikację w Safari.
  \item Kliknij ikonę \textbf{udostępniania} (prostokąt ze strzałką w górę).
  \item Wybierz \textbf{„Dodaj do ekranu głównego"}.
  \item Nadaj skrótowi nazwę i potwierdź.
\end{enumerate}

\subsection{Funkcje w trybie PWA}

Po zainstalowaniu aplikacja otwiera się \textbf{bez paska adresu przeglądarki},
obsługuje dolną nawigację z ikonami, dostosowuje układ do małych ekranów
oraz obsługuje gesty dotykowe.

\subsection{Logowanie Google w trybie standalone (iOS)}

Logowanie przez Google OAuth jest w pełni obsługiwane w trybie standalone.
Po kliknięciu „Zaloguj się przez Google" system otworzy stronę logowania Google
i po uwierzytelnieniu automatycznie powróci do aplikacji.

\begin{tipbox}
  \textbf{Uwaga:} Jeśli logowanie przez Google nie działa w trybie standalone,
  upewnij się, że otwierasz aplikację bezpośrednio z ikony na ekranie głównym.
  W razie problemów spróbuj usunąć aplikację z ekranu głównego i dodać ją
  ponownie.
\end{tipbox}

\subsection{Nawigacja mobilna}

Na urządzeniach mobilnych zamiast górnego paska pojawia się \textbf{dolna
nawigacja} z przyciskami:

\begin{longtable}{@{}C{2.5cm}C{3.5cm}L{9cm}@{}}
\toprule
\rowcolor{kenaznavy!8}
\textbf{Ikona} & \textbf{Strona} & \textbf{Opis} \\
\midrule
\endhead
\bottomrule
\endfoot
Dom       & \texttt{/}        & Strona główna \\
\rowcolor{tableshade}
Kalendarz & \texttt{/calendar} & Kalendarz wydarzeń \\
Chat      & \texttt{/chat}     & Chat i komentarze \\
\rowcolor{tableshade}
Panel     & \texttt{/panel}    & Twoje rejestracje \\
Konto     & \texttt{/me}       & Profil i ustawienia \\
\end{longtable}

\subsection{Przycisk feedbacku (żarówka)}

Na wszystkich stronach aplikacji dostępna jest \textbf{ruchoma ikonka żarówki}
w prawym dolnym rogu. Kliknięcie jej otwiera formularz feedbacku — można wpisać
opinię, sugestię lub zgłoszenie błędu. Działa poprawnie na wszystkich
urządzeniach.

% ──────────────────────────────────────────────────────────────────────────────
\section{Tryb ciemny i jasny}

\subsection{Jak zmienić tryb wyświetlania?}

Kenaz obsługuje dwa motywy kolorystyczne:

\begin{itemize}
  \item \textbf{Tryb jasny} — kremowe tło, granatowy tekst.
  \item \textbf{Tryb ciemny} — granatowe tło, kremowy tekst.
\end{itemize}

Możesz zmienić motyw klikając \textbf{ikonę słońca/księżyca} w górnym pasku
nawigacyjnym lub przez stronę \textbf{Moje konto} (\texttt{/me}).

\subsection{Automatyczne dopasowanie do systemu}

Przy pierwszej wizycie aplikacja \textbf{automatycznie} wykrywa preferencje
motywu ustawione w systemie operacyjnym.

\subsection{Zapamiętywanie preferencji}

Twój wybór jest zapisywany w przeglądarce (localStorage) i będzie zapamiętany
przy kolejnych wizytach.

% ──────────────────────────────────────────────────────────────────────────────
\section{Powiadomienia i komunikaty systemowe}

\subsection{Rodzaje powiadomień}

\begin{longtable}{@{}L{3.5cm}L{3.5cm}L{8cm}@{}}
\toprule
\rowcolor{kenaznavy!8}
\textbf{Typ} & \textbf{Wygląd} & \textbf{Kiedy się pojawia} \\
\midrule
\endhead
\bottomrule
\endfoot
\textbf{Sukces} (zielony)     & Zielony baner u góry   & Operacja zakończona pomyślnie \\
\rowcolor{tableshade}
\textbf{Błąd} (czerwony)      & Czerwony baner          & Coś poszło nie tak \\
\textbf{Informacja} (niebieski) & Niebieski baner        & Ogólna informacja \\
\rowcolor{tableshade}
\textbf{Potwierdzenie}        & Okno modalne            & Wymagane potwierdzenie akcji \\
\end{longtable}

\subsection{Banner powiadomień globalnych}

Administratorzy mogą ustawić \textbf{globalny baner informacyjny} pojawiający
się u góry strony dla wszystkich użytkowników (np.\ przerwa techniczna,
zmiana terminów).

\subsection{Powiadomienia o nowych wiadomościach}

Jeśli masz nieprzeczytane wiadomości w chacie, na ikonie chatu pojawia się
\textbf{czerwona kropka} lub liczba wiadomości. Po wejściu w wątek powiadomienie
znika.

% ──────────────────────────────────────────────────────────────────────────────
\section{Często zadawane pytania (FAQ)}

\textbf{P: Dlaczego nie mogę zobaczyć kalendarza?}

O: Musisz być zalogowany i mieć zatwierdzone konto przez administratora.
Jeśli się dopiero zarejestrowałeś, poczekaj na akceptację (1–2 dni robocze).

\bigskip
\textbf{P: Jak anulować subskrypcję?}

O: Przejdź do \texttt{/plans} i wybierz plan darmowy. Zmiana następuje
natychmiastowo.

\bigskip
\textbf{P: Zapłaciłem za wydarzenie, ale rejestracja nie działa — co robię?}

O: Sprawdź status płatności w panelu (\texttt{/panel}) lub na stronie
\texttt{/manual-payment/:id}. Upewnij się, że kliknąłeś „Potwierdzam
wykonanie przelewu". Jeśli problem nadal występuje, skontaktuj się
z administratorem.

\bigskip
\textbf{P: Nie mogę się zalogować przez Google — co robię?}

O: Sprawdź, czy Twoje połączenie internetowe działa. Spróbuj wyczyścić
cache przeglądarki lub użyć trybu prywatnego. Jeśli problem nie ustępuje,
skontaktuj się z administratorem.

\bigskip
\textbf{P: Czy mogę zmienić adres email w Kenaz?}

O: Kenaz używa konta Google, więc adres email jest powiązany z Twoim kontem
Google. Aby zmienić email, musisz użyć innego konta Google.

\bigskip
\textbf{P: Zapisałem się na listę oczekujących — co teraz?}

O: Poczekaj. System automatycznie awansuje Cię, gdy ktoś zwolni miejsce.
Sprawdzaj status w panelu (\texttt{/panel}).

\bigskip
\textbf{P: Jak wesprzeć Kenaz finansowo?}

O: Przejdź do \texttt{/support}. Możesz wykonać przelew na podany numer konta
lub skorzystać z zewnętrznych platform. Zalogowani użytkownicy mogą też
skorzystać z formularza darowizny w aplikacji.

\bigskip
\textbf{P: Czy mogę wspierać centrum bez zakładania konta?}

O: Tak. Strona \texttt{/support} jest dostępna publicznie.

\bigskip
\textbf{P: Nie mogę się zalogować przez Google w trybie PWA na iPhonie?}

O: Upewnij się, że otwierasz aplikację bezpośrednio z ikony na ekranie
głównym. Jeśli problem nadal występuje, spróbuj usunąć aplikację z ekranu
głównego i dodać ją ponownie przez Safari $\rightarrow$ Udostępnij
$\rightarrow$ Dodaj do ekranu głównego.

\bigskip
\textbf{P: Co oznacza „zablokowane konto"?}

O: Konto zablokowane przez administratora automatycznie wraca do statusu
oczekującego. Musisz poczekać, aż administrator ponownie je zatwierdzi.

% ──────────────────────────────────────────────────────────────────────────────
\clearpage
\section{Słownik pojęć}

\begin{longtable}{@{}L{4cm}L{11cm}@{}}
\toprule
\rowcolor{kenaznavy!8}
\textbf{Pojęcie} & \textbf{Definicja} \\
\midrule
\endhead
\bottomrule
\endfoot
\textbf{BANNED}              & Status konta zablokowanego przez administratora \\
\rowcolor{tableshade}
\textbf{Chat ogólny}         & Globalny kanał komunikacji dla wszystkich aktywnych użytkowników \\
\textbf{Darczyńca}           & Użytkownik, który przekazał darowiznę na rzecz organizacji \\
\rowcolor{tableshade}
\textbf{Feedback}            & Opinia lub sugestia wysłana przez żarówkę \\
\textbf{Limit uczestników}   & Maksymalna liczba osób zapisanych na wydarzenie \\
\rowcolor{tableshade}
\textbf{Lista oczekujących}  & Kolejka użytkowników czekających na zwolnienie miejsca \\
\textbf{Manualna płatność}   & Przelew bankowy weryfikowany ręcznie przez administratora \\
\rowcolor{tableshade}
\textbf{Motyw}               & Schemat kolorystyczny: jasny (light) lub ciemny (dark) \\
\textbf{PENDING}             & Status konta oczekującego na akceptację administratora \\
\rowcolor{tableshade}
\textbf{Plan darmowy}        & Podstawowy plan bez abonamentu \\
\textbf{Plan miesięczny}     & Subskrypcja odnawialna co miesiąc \\
\rowcolor{tableshade}
\textbf{Plan roczny}         & Subskrypcja na rok; najniższe ceny wydarzeń \\
\textbf{PWA}                 & Progressive Web App — aplikacja webowa działająca jak natywna \\
\rowcolor{tableshade}
\textbf{Reakcja}             & Emoji dodawany do wiadomości w chacie \\
\textbf{Rejestracja}         & Zapis użytkownika na konkretne wydarzenie \\
\rowcolor{tableshade}
\textbf{Rola}                & Uprawnienia: Guest (gość), Member (członek), Admin (administrator) \\
\textbf{Standalone}          & Tryb działania PWA na iOS bez paska adresu Safari \\
\rowcolor{tableshade}
\textbf{Status konta}        & Stan konta: Active, Pending lub Banned \\
\textbf{Subskrypcja}         & Płatny plan membership; niższe ceny na wydarzeniach \\
\rowcolor{tableshade}
\textbf{Tag zainteresowań}   & Etykieta określająca zainteresowania użytkownika \\
\textbf{Waitlist}            & Synonim listy oczekujących \\
\rowcolor{tableshade}
\textbf{Wątek}               & Zbiór odpowiedzi powiązanych z wiadomością w chacie \\
\textbf{Wesprzyj nas}        & Strona \texttt{/support} umożliwiająca wsparcie centrum \\
\end{longtable}

\vfill
\begin{center}
  {\color{kenaznavy!25}\rule{0.6\linewidth}{0.5pt}\par}
  \vspace{0.4cm}
  {\sffamily\small\color{kenazdim!70}
    Instrukcja Użytkownika — Kenaz Centrum, wersja 2.0, luty 2026.\par
    Wszelkie pytania kieruj do administratorów platformy.\par}
\end{center}

\end{document}
