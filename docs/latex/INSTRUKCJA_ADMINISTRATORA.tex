% ══════════════════════════════════════════════════════════════════════════════
%  INSTRUKCJA ADMINISTRATORA — Kenaz Centrum
%  Kompilacja: xelatex INSTRUKCJA_ADMINISTRATORA.tex  (dwukrotnie)
% ══════════════════════════════════════════════════════════════════════════════
\documentclass[11pt, a4paper]{article}

% ── Fonty ─────────────────────────────────────────────────────────────────────
\usepackage{fontspec}
\setmainfont{Palatino}[Ligatures=TeX, Numbers=OldStyle]
\setsansfont{Helvetica Neue}[BoldFont={Helvetica Neue Bold}, Scale=MatchLowercase]
\setmonofont{Courier New}[Scale=0.88]

% ── Język ─────────────────────────────────────────────────────────────────────
\usepackage{polyglossia}
\setdefaultlanguage{polish}

% ── Marginesy ─────────────────────────────────────────────────────────────────
\usepackage[a4paper, top=3cm, bottom=3cm, left=2.8cm, right=2.8cm, headheight=15pt]{geometry}

% ── Kolory ────────────────────────────────────────────────────────────────────
\usepackage{xcolor}
\definecolor{kenaznavy}{HTML}{0F174A}
\definecolor{kenazred}{HTML}{E53935}
\definecolor{kenazdim}{HTML}{4B5385}
\definecolor{tableshade}{HTML}{F5F4EF}
\definecolor{codebg}{HTML}{F0EFE8}
\definecolor{tipbg}{HTML}{EEF0F8}

% ── Hiperłącza ────────────────────────────────────────────────────────────────
\usepackage[unicode, colorlinks=true, linkcolor=kenaznavy,
            urlcolor=kenazred, citecolor=kenazdim,
            pdftitle={Instrukcja Administratora — Kenaz Centrum},
            pdfauthor={Kenaz Centrum}]{hyperref}

% ── Interlinia / akapity ──────────────────────────────────────────────────────
\usepackage{setspace}
\setstretch{1.4}
\setlength{\parskip}{6pt plus 2pt minus 1pt}
\setlength{\parindent}{0pt}

% ── Łamanie stron ─────────────────────────────────────────────────────────────
\widowpenalty=10000
\clubpenalty=10000
\displaywidowpenalty=10000

% ── Nagłówki i stopki ─────────────────────────────────────────────────────────
\usepackage{fancyhdr}
\pagestyle{fancy}
\fancyhf{}
\fancyhead[L]{\small\color{kenazdim}\nouppercase{\leftmark}}
\fancyhead[R]{\small\color{kenazdim}Kenaz Centrum}
\fancyfoot[C]{\small\color{kenazdim}---\;\thepage\;---}
\renewcommand{\headrulewidth}{0.3pt}
\renewcommand{\footrulewidth}{0pt}
\renewcommand{\headrule}{{\color{kenazdim!40}\hrule width\headwidth height 0.3pt}}
\fancypagestyle{plain}{%
  \fancyhf{}%
  \fancyfoot[C]{\small\color{kenazdim}---\;\thepage\;---}%
  \renewcommand{\headrulewidth}{0pt}%
}

% ── Nagłówki sekcji ───────────────────────────────────────────────────────────
\usepackage{titlesec}
\usepackage{needspace}
\titleformat{\section}
  {\sffamily\LARGE\bfseries\color{kenaznavy}}{\thesection}{0.8em}{}
\titleformat{\subsection}
  {\sffamily\large\bfseries\color{kenaznavy}}{\thesubsection}{0.8em}{}
\titleformat{\subsubsection}
  {\sffamily\normalsize\bfseries\color{kenazdim}}{\thesubsubsection}{0.8em}{}
\titlespacing{\section}     {0pt}{24pt plus 6pt minus 4pt}{8pt}
\titlespacing{\subsection}  {0pt}{16pt plus 4pt minus 2pt}{5pt}
\titlespacing{\subsubsection}{0pt}{10pt plus 2pt minus 2pt}{3pt}
\preto\section{\needspace{5\baselineskip}}
\preto\subsection{\needspace{4\baselineskip}}

% ── Spis treści ───────────────────────────────────────────────────────────────
\usepackage{tocloft}
\renewcommand{\cfttoctitlefont}{\sffamily\Large\bfseries\color{kenaznavy}}
\renewcommand{\cftsecfont}{\sffamily\bfseries\color{kenaznavy}}
\renewcommand{\cftsecpagefont}{\sffamily\small\color{kenazdim}}
\renewcommand{\cftsubsecfont}{\sffamily\small}
\renewcommand{\cftsubsecpagefont}{\sffamily\small\color{kenazdim}}
\setlength{\cftbeforesecskip}{3pt}

% ── Tabele ────────────────────────────────────────────────────────────────────
\usepackage{booktabs}
\usepackage{longtable}
\usepackage{array}
\usepackage{colortbl}
\usepackage{tabularx}
\newcolumntype{L}[1]{>{\raggedright\arraybackslash}p{#1}}
\newcolumntype{C}[1]{>{\centering\arraybackslash}p{#1}}
\arrayrulecolor{kenaznavy!20}
\renewcommand{\arraystretch}{1.3}

% ── Bloki kodu ────────────────────────────────────────────────────────────────
\usepackage{listings}
\lstset{
  basicstyle=\ttfamily\small,
  backgroundcolor=\color{codebg},
  frame=single,
  rulecolor=\color{kenaznavy!20},
  breaklines=true,
  breakatwhitespace=false,
  xleftmargin=10pt, xrightmargin=10pt,
  aboveskip=8pt, belowskip=8pt,
  showstringspaces=false,
}

% ── Ramki informacyjne (ostrzeżenia, wskazówki) ───────────────────────────────
\usepackage{tcolorbox}
\tcbuselibrary{skins, breakable}
\newtcolorbox{tipbox}{
  breakable,
  colback=tipbg,
  colframe=kenaznavy!35,
  boxrule=0.4pt,
  left=8pt, right=8pt, top=5pt, bottom=5pt,
  fontupper=\small,
  before upper={\setlength{\parskip}{3pt}},
}
\newtcolorbox{warnbox}{
  breakable,
  colback=kenazred!5,
  colframe=kenazred!40,
  boxrule=0.4pt,
  left=8pt, right=8pt, top=5pt, bottom=5pt,
  fontupper=\small,
  before upper={\setlength{\parskip}{3pt}},
}

% ── Listy ─────────────────────────────────────────────────────────────────────
\usepackage{enumitem}
\setlist[itemize]  {leftmargin=1.5em, itemsep=2pt, topsep=3pt, parsep=0pt}
\setlist[enumerate]{leftmargin=1.5em, itemsep=2pt, topsep=3pt, parsep=0pt}

% ── Mikrotypografia ───────────────────────────────────────────────────────────
\usepackage{microtype}

% ══════════════════════════════════════════════════════════════════════════════
\begin{document}

% ── Strona tytułowa ───────────────────────────────────────────────────────────
\thispagestyle{empty}
\begin{center}
  \vspace*{2.5cm}
  {\sffamily\fontsize{40}{48}\selectfont\bfseries\color{kenaznavy} Kenaz\par}
  \vspace{0.3cm}
  {\color{kenaznavy!25}\rule{\linewidth}{1.5pt}\par}
  \vspace{0.6cm}
  {\sffamily\Huge\bfseries\color{kenaznavy} Instrukcja Administratora\par}
  \vspace{0.4cm}
  {\sffamily\Large\color{kenazdim} Kompletny przewodnik zarządzania aplikacją\par}
  \vspace{1.2cm}
  {\color{kenaznavy!25}\rule{0.4\linewidth}{0.5pt}\par}
  \vspace{0.6cm}
  {\sffamily\normalsize\color{kenazdim}
    \textbf{Wersja dokumentu:} 2.0 \quad
    \textbf{Data:} luty 2026\par}
  \vspace{0.3cm}
  {\sffamily\small\color{kenazdim!80}
    Dotyczy: panel administracyjny aplikacji Kenaz\par}
  \vfill
  \begin{tcolorbox}[colback=kenazred!8, colframe=kenazred!50, boxrule=0.5pt,
                    left=12pt, right=12pt, top=8pt, bottom=8pt, width=0.9\linewidth]
    {\sffamily\small\color{kenazred!80}
      \textbf{Dokument poufny} — przeznaczony wyłącznie dla administratorów
      systemu. Nie udostępniaj go użytkownikom. Zakłada pełną znajomość
      aplikacji od strony użytkownika. Szczegółowy opis funkcji użytkownika
      zawiera osobny dokument: \emph{INSTRUKCJA\_UZYTKOWNIKA}.\par}
  \end{tcolorbox}
  \vspace{0.8cm}
  {\sffamily\small\color{kenazdim!60} Kenaz Centrum\par}
\end{center}
\clearpage

% ── Spis treści ───────────────────────────────────────────────────────────────
\tableofcontents
\clearpage

% ══════════════════════════════════════════════════════════════════════════════
\section{Dashboard administratora}
\label{sec:dashboard}

\textbf{Adres:} \texttt{/admin} \quad \textbf{Dostęp:} wyłącznie użytkownicy z rolą Admin

\subsection{Ogólne informacje}

Dashboard to \textbf{centralna strona panelu administracyjnego}. Jest to pierwszy
ekran, który zobaczysz po wejściu pod adres \texttt{/admin}. Na dashboardzie
wyświetlona jest \textbf{siatka kafelków}, z których każdy prowadzi do konkretnej
sekcji zarządzania.

\subsection{Siatka kafelków}

Każdy kafelek zawiera ikonę, tytuł i skrócony opis funkcji. Kliknięcie kafelka
przenosi do odpowiedniej podstrony. Dostępne kafelki:

\begin{longtable}{@{}L{4.5cm}L{5cm}L{5.5cm}@{}}
\toprule
\rowcolor{kenaznavy!8}
\textbf{Kafelek} & \textbf{Adres} & \textbf{Opis} \\
\midrule
\endhead
\bottomrule
\endfoot
\textbf{Utwórz wydarzenie}    & \texttt{/admin/create-event}    & Formularz tworzenia nowego wydarzenia \\
\rowcolor{tableshade}
\textbf{Zatwierdź użytkowników} & \texttt{/admin/users}          & Oczekujące konta do akceptacji \\
\textbf{Wszyscy użytkownicy}  & \texttt{/admin/all-users}       & Lista i zarządzanie wszystkimi kontami \\
\rowcolor{tableshade}
\textbf{Płatności online}     & \texttt{/admin/payments}        & Przegląd transakcji TPay \\
\textbf{Płatności manualne}   & \texttt{/admin/manual-payments} & Weryfikacja przelewów bankowych \\
\rowcolor{tableshade}
\textbf{Bilans}               & \texttt{/admin/balance}         & Przegląd finansowy organizacji \\
\textbf{Darowizny}            & \texttt{/admin/donations}       & Lista darowizn i konfiguracja \\
\rowcolor{tableshade}
\textbf{Feedback}             & \texttt{/admin/feedback}        & Opinie nadesłane przez użytkowników \\
\textbf{Ikonki}               & \texttt{/admin/icons}           & Zarządzanie ikonkami kategorii \\
\rowcolor{tableshade}
\textbf{Nadaj admina}         & \texttt{/admin/promote}         & Przyznawanie uprawnień administratora \\
\end{longtable}

\subsection{Widoczność dashboardu}

Dashboard i wszystkie podstrony \texttt{/admin/*} są dostępne \textbf{wyłącznie}
dla zalogowanych użytkowników z przypisaną rolą \texttt{Admin}. Próba wejścia na
te adresy przez zwykłego użytkownika skutkuje przekierowaniem do kalendarza lub
wyświetleniem komunikatu o braku uprawnień.

% ══════════════════════════════════════════════════════════════════════════════
\section{Tworzenie i edycja wydarzeń}
\label{sec:events}

\textbf{Adresy:} \texttt{/admin/create-event}, \texttt{/admin/edit-event/:id}
\quad \textbf{Dostęp:} wyłącznie Admin

\subsection{Jak utworzyć nowe wydarzenie?}

\begin{enumerate}
  \item Przejdź do dashboardu (\texttt{/admin}) i kliknij kafelek
        \textbf{„Utwórz wydarzenie"}, lub bezpośrednio wejdź pod adres
        \texttt{/admin/create-event}.
  \item Wypełnij formularz (wszystkie wymagane pola oznaczone gwiazdką \texttt{*}).
  \item Kliknij przycisk \textbf{„Utwórz wydarzenie"}.
  \item Wydarzenie od razu pojawia się w kalendarzu.
\end{enumerate}

\subsection{Pola formularza tworzenia wydarzenia}

\begin{longtable}{@{}L{3.8cm}C{2cm}L{9cm}@{}}
\toprule
\rowcolor{kenaznavy!8}
\textbf{Pole} & \textbf{Wymagane} & \textbf{Opis i uwagi} \\
\midrule
\endhead
\bottomrule
\endfoot
\textbf{Tytuł}             & Tak & Nazwa wyświetlana w kalendarzu i na stronie szczegółów \\
\rowcolor{tableshade}
\textbf{Data}              & Tak & Data i godzina rozpoczęcia; picker kalendarza \\
\textbf{Godzina zakończenia} & Nie & Opcjonalna; wyświetlana na stronie szczegółów \\
\rowcolor{tableshade}
\textbf{Miasto}            & Tak & Wybór z listy dostępnych miast; decyduje o filtrowaniu w kalendarzu \\
\textbf{Lokalizacja}       & Tak & Adres lub nazwa miejsca; wyświetlana z linkiem do Google Maps \\
\rowcolor{tableshade}
\textbf{Opis}              & Tak & Pełny opis aktywności; wspiera formatowanie tekstowe \\
\textbf{Cena (goście)}     & Tak & Cena dla użytkowników bez subskrypcji; wpisz \texttt{0} dla bezpłatnych \\
\rowcolor{tableshade}
\textbf{Cena (członkowie)} & Tak & Cena dla subskrybentów; wpisz \texttt{0} dla bezpłatnych \\
\textbf{Limit uczestników} & Nie & Maksymalna liczba miejsc; zostaw puste dla bez limitu \\
\rowcolor{tableshade}
\textbf{Ikona kategorii}   & Nie & Wybór ikony z zarządzanych typów \\
\textbf{Ogłoszenie}        & Nie & Opcjonalny anons powiązany z wydarzeniem \\
\end{longtable}

\subsection{Limit wydarzeń dziennie}

System nakłada ograniczenie: \textbf{maksymalnie 4 wydarzenia w jednym dniu
w tym samym mieście}. Próba przekroczenia limitu wyświetla komunikat błędu
i uniemożliwia zapisanie.

\subsection{Edycja istniejącego wydarzenia}

\begin{enumerate}
  \item Wejdź na stronę szczegółów danego wydarzenia (\texttt{/event/:id}).
  \item Kliknij przycisk \textbf{„Edytuj"} (widoczny tylko dla administratorów).
  \item Zostaniesz przekierowany do formularza edycji pod adresem
        \texttt{/admin/edit-event/:id}.
  \item Zmień wybrane pola i kliknij \textbf{„Zapisz zmiany"}.
\end{enumerate}

Edycja jest natychmiastowa — zmiany są widoczne dla wszystkich użytkowników
od razu po zapisaniu.

\subsection{Usuwanie wydarzenia}

Na stronie edycji wydarzenia (\texttt{/admin/edit-event/:id}) dostępny jest
przycisk \textbf{„Usuń wydarzenie"}. Usunięcie jest \textbf{nieodwracalne}
i wiąże się z anulowaniem wszystkich rejestracji na to wydarzenie.

\begin{warnbox}
  \textbf{Ważne:} Przed usunięciem przejrzyj listę uczestników. Jeśli wydarzenie
  ma zarejestrowanych uczestników, którzy zapłacili, rozważ poinformowanie ich
  przed usunięciem.
\end{warnbox}

% ══════════════════════════════════════════════════════════════════════════════
\section{Zatwierdzanie nowych użytkowników}
\label{sec:approve}

\textbf{Adres:} \texttt{/admin/users} \quad \textbf{Dostęp:} wyłącznie Admin

\subsection{Co widać na tej stronie?}

Na stronie \texttt{/admin/users} wyświetlana jest \textbf{lista kont oczekujących
na zatwierdzenie}. Każda karta użytkownika zawiera:

\begin{itemize}
  \item \textbf{Imię i nazwisko} (z konta Google).
  \item \textbf{Adres email}.
  \item \textbf{Awatar} (jeśli dostępny z Google).
  \item \textbf{Data rejestracji} — kiedy użytkownik zalogował się po raz pierwszy.
  \item \textbf{Przycisk „Zatwierdź"} i \textbf{przycisk „Odrzuć"}.
\end{itemize}

\subsection{Jak zatwierdzić konto?}

\begin{enumerate}
  \item Przejdź do \texttt{/admin/users}.
  \item Znajdź kartę użytkownika, którego chcesz zatwierdzić.
  \item Kliknij przycisk \textbf{„Zatwierdź"}.
  \item Status konta zmienia się z \texttt{PENDING} na \texttt{ACTIVE}.
  \item Użytkownik przy kolejnym wejściu w aplikację zobaczy ekran powitalny
        i wybór planu subskrypcji.
\end{enumerate}

\begin{tipbox}
  \textbf{Wskazówka:} Rozważ weryfikację tożsamości użytkownika przed
  zatwierdzeniem (jeśli masz takie procedury). Po zatwierdzeniu użytkownik
  uzyskuje pełny dostęp do wszystkich funkcji.
\end{tipbox}

\subsection{Co się dzieje po zatwierdzeniu?}

\begin{itemize}
  \item Konto zmienia status na \texttt{ACTIVE}.
  \item Użytkownik zostaje przekierowany do ekranu powitalnego z wyborem planu
        subskrypcji przy kolejnym logowaniu.
  \item Zostaje zalogowane zdarzenie w systemie logów audytowych.
\end{itemize}

\subsection{Jeśli lista jest pusta}

Gdy nie ma oczekujących kont, strona wyświetla komunikat informujący, że nie ma
nic do zatwierdzenia. Jest to normalna sytuacja — wróć tu, gdy pojawią się
nowi użytkownicy.

% ══════════════════════════════════════════════════════════════════════════════
\section{Lista wszystkich użytkowników}
\label{sec:allusers}

\textbf{Adres:} \texttt{/admin/all-users} \quad \textbf{Dostęp:} wyłącznie Admin

\subsection{Ogólny opis}

Strona \texttt{/admin/all-users} to \textbf{kompletna lista wszystkich kont}
w systemie Kenaz, bez względu na status. Służy do zarządzania aktywnymi,
oczekującymi i zablokowanymi użytkownikami.

\subsection{Paginacja}

Użytkownicy są wyświetlani stronicowo. Na dole listy znajdziesz przyciski
nawigacji \textbf{„Następna strona"} / \textbf{„Poprzednia strona"} oraz
informację o aktualnej stronie i całkowitej liczbie użytkowników.

\subsection{Filtry}

\begin{longtable}{@{}L{4.5cm}L{10.5cm}@{}}
\toprule
\rowcolor{kenaznavy!8}
\textbf{Filtr} & \textbf{Opis} \\
\midrule
\endhead
\bottomrule
\endfoot
\textbf{Filtr statusu}       & Pokaż wszystkich / tylko aktywnych / tylko oczekujących / tylko zablokowanych \\
\rowcolor{tableshade}
\textbf{Filtr subskrybentów} & Pokaż wszystkich / tylko subskrybentów (aktywna subskrypcja płatna) \\
\textbf{Wyszukiwarka}        & Wyszukaj po imieniu lub adresie email \\
\end{longtable}

\subsection{Klikalne wiersze}

Każdy wiersz tabeli użytkowników jest \textbf{klikalny}. Kliknięcie otwiera
\textbf{kafelek admina} — rozszerzony panel informacyjny z pełnymi danymi
o danym koncie.

\subsection{Kafelek admina — zawartość}

Kafelek admina jest powiększoną kartą informacyjną podzieloną na cztery sekcje:

\subsubsection{Sekcja 1: Konto}

\begin{longtable}{@{}L{4cm}L{11cm}@{}}
\toprule
\rowcolor{kenaznavy!8}
\textbf{Pole} & \textbf{Opis} \\
\midrule
\endhead
\bottomrule
\endfoot
\textbf{Imię i email}    & Dane identyfikacyjne \\
\rowcolor{tableshade}
\textbf{Awatar}          & Zdjęcie z Google \\
\textbf{Rola}            & Guest, Member lub Admin \\
\rowcolor{tableshade}
\textbf{Status konta}    & Active, Pending lub Banned \\
\textbf{Data rejestracji} & Kiedy użytkownik zalogował się po raz pierwszy \\
\rowcolor{tableshade}
\textbf{Miasto}          & Wybrane miasto użytkownika \\
\end{longtable}

\subsubsection{Sekcja 2: Aktywność}

\begin{longtable}{@{}L{4cm}L{11cm}@{}}
\toprule
\rowcolor{kenaznavy!8}
\textbf{Pole} & \textbf{Opis} \\
\midrule
\endhead
\bottomrule
\endfoot
\textbf{Liczba rejestracji} & Ile razy użytkownik zapisał się na wydarzenia \\
\rowcolor{tableshade}
\textbf{Ostatnia aktywność} & Data ostatniego logowania lub akcji \\
\textbf{Aktywny subskrybent} & Tak/Nie; nazwa planu i data wygaśnięcia \\
\end{longtable}

\subsubsection{Sekcja 3: Finanse}

\begin{longtable}{@{}L{4cm}L{11cm}@{}}
\toprule
\rowcolor{kenaznavy!8}
\textbf{Pole} & \textbf{Opis} \\
\midrule
\endhead
\bottomrule
\endfoot
\textbf{Suma wpłat}        & Łączna kwota zapłacona za wydarzenia \\
\rowcolor{tableshade}
\textbf{Suma subskrypcji}  & Łączna kwota zapłacona za subskrypcje \\
\textbf{Darowizny}         & Łączna kwota darowizn \\
\rowcolor{tableshade}
\textbf{Historia transakcji} & Lista ostatnich płatności \\
\end{longtable}

\subsubsection{Sekcja 4: Oczekujące akcje}

\begin{longtable}{@{}L{4cm}L{11cm}@{}}
\toprule
\rowcolor{kenaznavy!8}
\textbf{Pole} & \textbf{Opis} \\
\midrule
\endhead
\bottomrule
\endfoot
\textbf{Oczekujące płatności}  & Rejestracje, za które użytkownik jeszcze nie zapłacił \\
\rowcolor{tableshade}
\textbf{Oczekujące rejestracje} & Rejestracje do potwierdzenia \\
\end{longtable}

\subsection{Blokowanie i odblokowanie konta}

Z poziomu kafelka admina możesz zarządzać statusem konta.

\textbf{Zablokowanie konta (\texttt{ACTIVE} $\rightarrow$ \texttt{BANNED}):}

\begin{enumerate}
  \item W kafelku admina kliknij \textbf{„Zablokuj konto"}.
  \item Pojawi się okno potwierdzenia — kliknij \textbf{„Potwierdź"}.
  \item Konto zmienia status na \texttt{BANNED}. Użytkownik traci dostęp
        do pełnych funkcji.
\end{enumerate}

\textbf{Odblokowanie konta (\texttt{BANNED} $\rightarrow$ \texttt{PENDING} /
\texttt{ACTIVE}):}

\begin{enumerate}
  \item W kafelku admina przy zablokowanym koncie kliknij \textbf{„Odblokuj"}.
  \item Konto wraca do statusu \texttt{PENDING} (wymaga ponownej akceptacji)
        lub \texttt{ACTIVE} (bezpośrednie odblokowanie), zależnie od konfiguracji.
\end{enumerate}

\begin{tipbox}
  \textbf{Uwaga:} Zablokowany użytkownik widzi kalendarz i panel zamazane,
  identycznie jak przy statusie PENDING. Jego rejestracje i historia pozostają
  w systemie.
\end{tipbox}

% ══════════════════════════════════════════════════════════════════════════════
\section{Płatności online}
\label{sec:payments-online}

\textbf{Adres:} \texttt{/admin/payments} \quad \textbf{Dostęp:} wyłącznie Admin

\subsection{Co widać na tej stronie?}

Strona \texttt{/admin/payments} wyświetla \textbf{listę transakcji przetwarzanych
przez bramkę płatności online} (TPay). Każda pozycja na liście zawiera:

\begin{itemize}
  \item \textbf{ID transakcji} — unikalny identyfikator z systemu płatności.
  \item \textbf{Użytkownik} — imię i email płacącego.
  \item \textbf{Kwota} — wartość transakcji.
  \item \textbf{Status transakcji} — np.\ Opłacona, Oczekująca, Anulowana,
        Zwrócona.
  \item \textbf{Data i godzina} transakcji.
  \item \textbf{Typ płatności} — za wydarzenie, za subskrypcję.
\end{itemize}

\subsection{Ręczna zmiana statusu transakcji}

W uzasadnionych przypadkach (np.\ bramka nie zaktualizowała statusu automatycznie)
możesz \textbf{ręcznie zmienić status transakcji}:

\begin{enumerate}
  \item Kliknij w wybraną transakcję na liście.
  \item Wybierz nowy status z listy dostępnych opcji.
  \item Kliknij \textbf{„Zapisz"}.
\end{enumerate}

\begin{warnbox}
  \textbf{Ważne:} Ręczna zmiana statusu powinna być stosowana wyłącznie
  w wyjątkowych sytuacjach — np.\ gdy bramka płatności zgłasza błąd, ale
  wpłata faktycznie dotarła. Każda zmiana jest odnotowywana w logach
  audytowych i widoczna przez system.
\end{warnbox}

% ══════════════════════════════════════════════════════════════════════════════
\section{Płatności manualne}
\label{sec:payments-manual}

\textbf{Adres:} \texttt{/admin/manual-payments} \quad \textbf{Dostęp:}
wyłącznie Admin

\subsection{Co to są płatności manualne?}

Płatności manualne to \textbf{przelewy bankowe} dokonywane przez użytkowników
za rejestracje na płatne wydarzenia lub zakup subskrypcji. Administrator musi
ręcznie zweryfikować każdą płatność i ją zatwierdzić lub odrzucić.

\subsection{Główna lista — oczekujące płatności}

Na górze strony wyświetlona jest lista \textbf{oczekujących płatności do
weryfikacji}. Każda pozycja zawiera:

\begin{itemize}
  \item \textbf{Imię i email} użytkownika.
  \item \textbf{Typ płatności} — za wydarzenie lub za subskrypcję.
  \item \textbf{Kwota} do weryfikacji.
  \item \textbf{Data} złożenia prośby o potwierdzenie przez użytkownika.
  \item \textbf{Przycisk „Zweryfikuj"} i \textbf{przycisk „Odrzuć"}.
\end{itemize}

\subsection{Jak zweryfikować płatność?}

\begin{enumerate}
  \item Znajdź w banku przelew od danego użytkownika. Zweryfikuj kwotę
        i tytuł przelewu.
  \item Na stronie \texttt{/admin/manual-payments} kliknij \textbf{„Zweryfikuj"}
        przy odpowiedniej pozycji.
  \item Pojawi się okno potwierdzenia.
  \item Kliknij \textbf{„Potwierdź"} — rejestracja lub subskrypcja zostaje
        aktywowana.
\end{enumerate}

\subsection{Jak odrzucić płatność?}

Jeśli przelew nie dotarł, ma nieprawidłową kwotę lub jest niepoprawny
z innego powodu:

\begin{enumerate}
  \item Kliknij \textbf{„Odrzuć"} przy wybranej pozycji.
  \item Wpisz opcjonalnie powód odrzucenia (wyświetlany użytkownikowi).
  \item Kliknij \textbf{„Potwierdź odrzucenie"}.
  \item Rejestracja wraca do statusu \textbf{„Oczekuje na płatność"} —
        użytkownik musi wykonać przelew ponownie.
\end{enumerate}

\subsection{Filtry}

\begin{longtable}{@{}L{4cm}L{11cm}@{}}
\toprule
\rowcolor{kenaznavy!8}
\textbf{Filtr} & \textbf{Opis} \\
\midrule
\endhead
\bottomrule
\endfoot
\textbf{Typ}         & Tylko płatności za wydarzenia / tylko za subskrypcje / wszystkie \\
\rowcolor{tableshade}
\textbf{Status}      & Oczekujące / Rozpatrzone / Wszystkie \\
\textbf{Zakres dat}  & Filtruj płatności z wybranego okresu \\
\end{longtable}

\subsection{Sekcja „Rozpatrzone"}

Pod sekcją oczekujących płatności dostępna jest \textbf{historia rozpatrzonych
płatności}. Zawiera zarówno zatwierdzone, jak i odrzucone transakcje. Służy do
audytu i ewentualnego odwołania się do historii decyzji.

% ══════════════════════════════════════════════════════════════════════════════
\section{Bilans finansowy}
\label{sec:balance}

\textbf{Adres:} \texttt{/admin/balance} \quad \textbf{Dostęp:} wyłącznie Admin

\subsection{Ogólny przegląd}

Strona \texttt{/admin/balance} to \textbf{centrum analizy finansowej} organizacji.
Dostarcza zbiorczych informacji o przychodach z różnych źródeł.

\subsection{Sekcje raportu finansowego}

\begin{longtable}{@{}L{5cm}L{10cm}@{}}
\toprule
\rowcolor{kenaznavy!8}
\textbf{Sekcja} & \textbf{Opis} \\
\midrule
\endhead
\bottomrule
\endfoot
\textbf{Subskrypcje}             & Łączny przychód ze sprzedanych planów \\
\rowcolor{tableshade}
\textbf{Rejestracje na wydarzenia} & Łączny przychód z płatnych wydarzeń \\
\textbf{Darowizny}               & Łączna kwota wpłaconych darowizn \\
\rowcolor{tableshade}
\textbf{Suma całkowita}          & Łączny przychód ze wszystkich źródeł \\
\end{longtable}

Każda sekcja wyświetla kwotę za bieżący miesiąc, bieżący rok oraz łącznie
od początku działania systemu.

\subsection{Eksport danych}

Na stronie dostępny jest przycisk \textbf{„Eksportuj"} umożliwiający pobranie
danych finansowych w formacie CSV, co umożliwia dalszą analizę w arkuszu
kalkulacyjnym.

\subsection{Ręczne korekty bilansu}

W sekcji \textbf{„Korekty manualne"} administrator może dodać ręczną pozycję
finansową:

\begin{enumerate}
  \item Kliknij \textbf{„Dodaj korektę"}.
  \item Wpisz \textbf{tytuł korekty}, \textbf{kwotę} (dodatnią lub ujemną)
        i \textbf{datę}.
  \item Kliknij \textbf{„Zapisz"}.
  \item Korekta pojawia się w historii i wpływa na łączne saldo.
\end{enumerate}

\begin{tipbox}
  \textbf{Ważne:} Korekty manualne są wyłącznie zapisem ewidencyjnym —
  nie generują faktycznych transakcji bankowych.
\end{tipbox}

% ══════════════════════════════════════════════════════════════════════════════
\section{Darowizny}
\label{sec:donations}

\textbf{Adres:} \texttt{/admin/donations} \quad \textbf{Dostęp:} wyłącznie Admin

\subsection{Lista darowizn}

Strona \texttt{/admin/donations} wyświetla \textbf{listę wszystkich darowizn}
zarówno złożonych przez formularz w aplikacji, jak i zweryfikowanych manualnych
przelewów. Każda pozycja zawiera:

\begin{itemize}
  \item \textbf{Imię i email} darczyńcy.
  \item \textbf{Kwotę} darowizny.
  \item \textbf{Datę} wpłaty.
  \item \textbf{Status} — Zweryfikowana / Oczekuje na weryfikację.
  \item \textbf{Notatkę} od darczyńcy (jeśli podana).
\end{itemize}

\subsection{Konfiguracja konta bankowego}

W sekcji \textbf{„Konfiguracja"} możesz ustawić dane konta bankowego
wyświetlane użytkownikom na stronie \texttt{/support}:

\begin{enumerate}
  \item Kliknij \textbf{„Edytuj dane bankowe"}.
  \item Wpisz lub zaktualizuj \textbf{numer konta (IBAN)}, \textbf{nazwę
        odbiorcy} i \textbf{tytuł przelewu} — wzorzec tytułu dla darczyńców.
  \item Kliknij \textbf{„Zapisz"}.
\end{enumerate}

Zmiany są natychmiastowe — nowe dane pojawią się na stronie \texttt{/support}
przy kolejnym załadowaniu.

\subsection{Konfiguracja zewnętrznych linków}

W tej samej sekcji możesz zarządzać \textbf{przyciskami do zewnętrznych
platform} (np.\ buycoffee.to, Patronite):

\begin{enumerate}
  \item Kliknij \textbf{„Dodaj link zewnętrzny"} lub edytuj istniejący.
  \item Wpisz \textbf{nazwę platformy} i \textbf{URL} (pełny adres strony
        profilowej Kenaz).
  \item Kliknij \textbf{„Zapisz"}.
\end{enumerate}

Linki pojawią się jako przyciski na stronie \texttt{/support}.

% ══════════════════════════════════════════════════════════════════════════════
\section{Opinie i feedback}
\label{sec:feedback}

\textbf{Adres:} \texttt{/admin/feedback} \quad \textbf{Dostęp:} wyłącznie Admin

\subsection{Skąd pochodzi feedback?}

Użytkownicy mogą wysyłać opinie i zgłoszenia przez \textbf{ikonkę żarówki}
dostępną w prawym dolnym rogu wszystkich stron aplikacji. Każdy feedback trafia
do panelu \texttt{/admin/feedback}.

\subsection{Zawartość wpisu feedback}

Każda opinia wyświetla:

\begin{itemize}
  \item \textbf{Treść wiadomości} — pełna treść przesłana przez użytkownika.
  \item \textbf{Data i godzina} wysłania.
  \item \textbf{Email użytkownika} (jeśli był zalogowany podczas wysyłania).
  \item \textbf{URL strony} — z jakiej strony aplikacji wysłano feedback.
\end{itemize}

\subsection{Zarządzanie feedbackiem}

\begin{longtable}{@{}L{4.5cm}L{10.5cm}@{}}
\toprule
\rowcolor{kenaznavy!8}
\textbf{Akcja} & \textbf{Opis} \\
\midrule
\endhead
\bottomrule
\endfoot
\textbf{Oznacz jako przeczytane} & Ukrywa powiadomienie; feedback pozostaje w historii \\
\rowcolor{tableshade}
\textbf{Archiwizuj}              & Przenosi do archiwum (dostępne przez filtr) \\
\textbf{Usuń}                    & Trwale usuwa feedback z systemu \\
\end{longtable}

\subsection{Filtry}

\begin{longtable}{@{}L{4cm}L{11cm}@{}}
\toprule
\rowcolor{kenaznavy!8}
\textbf{Filtr} & \textbf{Opis} \\
\midrule
\endhead
\bottomrule
\endfoot
\textbf{Status}      & Nowe / Przeczytane / Zarchiwizowane / Wszystkie \\
\rowcolor{tableshade}
\textbf{Zakres dat}  & Filtruj po dacie wysłania \\
\end{longtable}

% ══════════════════════════════════════════════════════════════════════════════
\section{Zarządzanie ikonkami wydarzeń}
\label{sec:icons}

\textbf{Adres:} \texttt{/admin/icons} \quad \textbf{Dostęp:} wyłącznie Admin

\subsection{Do czego służą ikonki?}

Ikonki to \textbf{kategorie wizualne} przypisywane do wydarzeń. Ułatwiają
użytkownikom szybkie rozpoznanie rodzaju aktywności na kalendarzu i stronie
szczegółów.

\subsection{Wbudowane typy ikonek}

System posiada 10 wbudowanych typów ikonek:

\begin{longtable}{@{}L{4cm}L{5cm}L{5cm}@{}}
\toprule
\rowcolor{kenaznavy!8}
\textbf{Nazwa} & \textbf{Symbol} & \textbf{Kolor} \\
\midrule
\endhead
\bottomrule
\endfoot
Karate        & karate    & Czerwony    \\
\rowcolor{tableshade}
Morsowanie    & lodowiec  & Niebieski   \\
Basen         & basen     & Niebieski   \\
\rowcolor{tableshade}
Board game    & gra       & Zielony     \\
Yoga          & yoga      & Fioletowy   \\
\rowcolor{tableshade}
Nordic Walking & spacer   & Zielony     \\
Wycieczka     & wycieczka & Brązowy     \\
\rowcolor{tableshade}
Siłownia      & siłownia  & Szary       \\
Gotowanie     & gotowanie & Pomarańczowy \\
\rowcolor{tableshade}
Inne          & inne      & Szary       \\
\end{longtable}

\subsection{Niestandardowe typy ikonek}

Oprócz wbudowanych typów możesz tworzyć \textbf{własne typy ikonek}
przechowywane w bazie danych. Ikonki te mogą być przypisywane do wydarzeń
identycznie jak wbudowane.

\subsection{Formularz tworzenia ikonki}

\begin{enumerate}
  \item Na stronie \texttt{/admin/icons} znajdź formularz \textbf{„Dodaj
        nową ikonkę"}.
  \item Wypełnij: \textbf{Nazwa}, \textbf{Emoji} i \textbf{Kolor} (палитра
        lub wartość HEX).
  \item Obserwuj \textbf{podgląd na żywo} — na bieżąco pokazuje, jak ikona
        będzie wyglądać.
  \item Kliknij \textbf{„Dodaj ikonkę"}.
\end{enumerate}

\subsection{Edycja i usuwanie ikonek}

Przy każdej niestandardowej ikonki dostępne są przyciski:

\begin{itemize}
  \item \textbf{Edytuj} — otwiera formularz edycji z aktualnymi danymi.
  \item \textbf{Usuń} — trwale usuwa ikonkę. Jeśli ikonka jest przypisana
        do wydarzeń, zostaje z nich usunięta.
\end{itemize}

\begin{tipbox}
  \textbf{Uwaga:} Wbudowanych 10 ikonek nie można usunąć — można je
  tylko wyłączyć (jeśli taka opcja jest dostępna).
\end{tipbox}

% ══════════════════════════════════════════════════════════════════════════════
\section{Nadawanie uprawnień administratora}
\label{sec:promote}

\textbf{Adres:} \texttt{/admin/promote} \quad \textbf{Dostęp:} wyłącznie Admin

\subsection{Jak nadać uprawnienia administratora?}

Dodanie nowego administratora to \textbf{operacja krytyczna} wymagająca
potwierdzenia kodem.

\textbf{Krok 1: Wpisz email użytkownika}

\begin{enumerate}
  \item Przejdź do \texttt{/admin/promote}.
  \item Wpisz \textbf{adres email} użytkownika, któremu chcesz nadać
        uprawnienia administratora.
  \item Upewnij się, że użytkownik ma aktywne konto w systemie.
\end{enumerate}

\textbf{Krok 2: Przeczytaj ostrzeżenie}

System wyświetli \textbf{ostrzeżenie} informujące o konsekwencjach nadania
uprawnień. Administrator ma pełny dostęp do panelu, danych użytkowników,
finansów oraz może nadawać uprawnienia innym.

\textbf{Krok 3: Wpisz kod potwierdzający}

Aby potwierdzić operację, wpisz w wyznaczone pole \textbf{kod potwierdzający}
wyświetlony na ekranie. Kod jest generowany losowo przy każdej próbie nadania
uprawnień i służy jako zabezpieczenie przed przypadkowym kliknięciem.

\textbf{Krok 4: Potwierdź}

Kliknij przycisk \textbf{„Nadaj uprawnienia"}. Jeśli kod jest poprawny,
uprawnienia zostaną przyznane.

\subsection{Efekty nadania uprawnień}

\begin{itemize}
  \item Użytkownik zmienia rolę z \texttt{Member} na \texttt{Admin}.
  \item Zmiana jest natychmiastowa — przy kolejnym zalogowaniu użytkownik
        zobaczy panel administracyjny.
  \item Zdarzenie zostaje zalogowane w systemie logów audytowych.
\end{itemize}

\subsection{Odbieranie uprawnień administratora}

Odebranie uprawnień administratora \textbf{nie jest dostępne przez interfejs
graficzny}. W razie potrzeby należy dokonać zmiany bezpośrednio w bazie danych
lub przez terminal SSH na serwerze.

\begin{warnbox}
  \textbf{Ważne:} Zachowaj ostrożność przy nadawaniu uprawnień administratora.
  Każda osoba z tą rolą ma pełny dostęp do danych użytkowników, historii
  płatności i konfiguracji systemu.
\end{warnbox}

% ══════════════════════════════════════════════════════════════════════════════
\section{Logi audytowe}
\label{sec:logs}

\textbf{Dostęp:} przez SSH na serwer

\subsection{Czym są logi audytowe?}

System logowania automatycznie zapisuje \textbf{kluczowe zdarzenia} w aplikacji
— logowania, rejestracje, płatności, akcje administratorów i błędy. Logi służą
do audytu bezpieczeństwa, debugowania i śledzenia aktywności.

\subsection{Lokalizacja logów na serwerze}

Logi są przechowywane na serwerze AWS EC2 w katalogu:

\begin{lstlisting}
/opt/kenaz/logs/
\end{lstlisting}

Struktura katalogów:

\begin{lstlisting}
logs/
+-- DD-MM-YYYY/
    +-- user@example.com.log
    +-- admin@kenaz.pl.log
    +-- system.log
\end{lstlisting}

Każdy dzień ma \textbf{własny podkatalog} z datą (format \texttt{DD-MM-YYYY}).
W katalogu danego dnia znajdują się pliki logów podzielone według adresu email
użytkownika.

\subsection{Format logów}

Każda linia logu ma format:

\begin{lstlisting}
[HH:MM:SS] POZIOM - tresc komunikatu
\end{lstlisting}

Przykład:

\begin{lstlisting}
[14:23:05] INFO    - User jan.kowalski@gmail.com logged in
[14:25:10] INFO    - Registration created: user=jan.kowalski@gmail.com event_id=42
[14:26:00] WARNING - Payment verification failed: registration_id=123
[15:00:00] ERROR   - Database connection timeout
\end{lstlisting}

\subsection{Poziomy logowania}

\begin{longtable}{@{}L{2.5cm}L{2.5cm}L{10cm}@{}}
\toprule
\rowcolor{kenaznavy!8}
\textbf{Poziom} & \textbf{Kolor} & \textbf{Kiedy używany} \\
\midrule
\endhead
\bottomrule
\endfoot
\texttt{INFO}    & Biały    & Normalne zdarzenia (logowanie, rejestracja) \\
\rowcolor{tableshade}
\texttt{WARNING} & Żółty    & Podejrzane zdarzenia lub nieudane operacje \\
\texttt{ERROR}   & Czerwony & Krytyczne błędy wymagające uwagi \\
\rowcolor{tableshade}
\texttt{DEBUG}   & Szary    & Szczegółowe informacje techniczne (tylko tryb dev) \\
\end{longtable}

\subsection{Rejestrowane zdarzenia}

\begin{longtable}{@{}L{5.5cm}C{2.5cm}L{7cm}@{}}
\toprule
\rowcolor{kenaznavy!8}
\textbf{Zdarzenie} & \textbf{Poziom} & \textbf{Opis wpisu} \\
\midrule
\endhead
\bottomrule
\endfoot
Logowanie użytkownika           & INFO    & Email + timestamp \\
\rowcolor{tableshade}
Wylogowanie                     & INFO    & Email + timestamp \\
Rejestracja na wydarzenie       & INFO    & Email + ID wydarzenia \\
\rowcolor{tableshade}
Anulowanie rejestracji          & INFO    & Email + ID rejestracji \\
Potwierdzenie płatności (użytk.)& INFO    & Email + kwota + ID rejestracji \\
\rowcolor{tableshade}
Weryfikacja płatności (admin)   & INFO    & Email admina + email użytk. + ID \\
Odrzucenie płatności (admin)    & WARNING & Email admina + powód \\
\rowcolor{tableshade}
Zablokowanie konta              & WARNING & Email admina + email użytkownika \\
Nadanie uprawnień admina        & WARNING & Email admina + email promowanego \\
\rowcolor{tableshade}
Nieudane logowanie              & WARNING & IP + email \\
Błąd systemu płatności          & ERROR   & Szczegóły błędu \\
\rowcolor{tableshade}
Błąd bazy danych                & ERROR   & Treść błędu SQL \\
Nieautoryzowany dostęp do panelu& WARNING & IP + URL \\
\end{longtable}

\subsection{Jak przeglądać logi — polecenia SSH}

Połącz się z serwerem przez SSH:

\begin{lstlisting}[language=bash]
ssh -i .secrets/KenazKeySSH.pem ec2-user@35.157.165.112
\end{lstlisting}

Logi z bieżącego dnia:

\begin{lstlisting}[language=bash]
ls /opt/kenaz/logs/$(date +%d-%m-%Y)/
\end{lstlisting}

Logi konkretnego użytkownika (bieżący dzień):

\begin{lstlisting}[language=bash]
cat /opt/kenaz/logs/$(date +%d-%m-%Y)/jan.kowalski@gmail.com.log
\end{lstlisting}

Logi systemowe z konkretnej daty:

\begin{lstlisting}[language=bash]
cat /opt/kenaz/logs/15-01-2026/system.log
\end{lstlisting}

Filtrowanie błędów ze wszystkich plików z bieżącego dnia:

\begin{lstlisting}[language=bash]
grep "ERROR" /opt/kenaz/logs/$(date +%d-%m-%Y)/*.log
\end{lstlisting}

Podgląd logów w czasie rzeczywistym:

\begin{lstlisting}[language=bash]
tail -f /opt/kenaz/logs/$(date +%d-%m-%Y)/system.log
\end{lstlisting}

Wyszukiwanie po emailu przez wiele dni:

\begin{lstlisting}[language=bash]
grep "jan.kowalski@gmail.com" /opt/kenaz/logs/**/*.log
\end{lstlisting}

\subsection{Retencja logów}

Logi są przechowywane bezterminowo, dopóki nie zostaną ręcznie usunięte lub
nie skończy się miejsce na dysku. Zalecamy \textbf{cykliczne archiwizowanie}
starszych logów (np.\ starszych niż 90 dni).

\subsection{Uprawnienia do logów}

Pliki logów są własnością użytkownika systemowego \texttt{ec2-user}. Dostęp
do logów mają wyłącznie osoby posiadające klucz SSH. Logów \textbf{nie można}
przeglądać przez interfejs webowy aplikacji.

% ══════════════════════════════════════════════════════════════════════════════
\clearpage
\section*{Często zadawane pytania (FAQ)}
\addcontentsline{toc}{section}{Często zadawane pytania (FAQ)}
\markboth{Często zadawane pytania (FAQ)}{}

\textbf{P: Użytkownik twierdzi, że zapłacił, ale płatność nie widnieje
w systemie — co robię?}

O: Sprawdź panel płatności manualnych (\texttt{/admin/manual-payments}).
Upewnij się, że użytkownik kliknął „Potwierdzam wykonanie przelewu"
w aplikacji. Jeśli nie, poinstruuj go. Następnie sprawdź konto bankowe
— jeśli wpłata dotarła, zatwierdź manualnie.

\bigskip
\textbf{P: Jak cofnąć zatwierdzenie konta użytkownika?}

O: Możesz \textbf{zablokować} konto użytkownika z poziomu \texttt{/admin/all-users}
(kliknij w użytkownika → „Zablokuj konto"). Konto wraca do statusu BANNED
i wymaga ponownej akceptacji.

\bigskip
\textbf{P: Admin usunął wydarzenie przez pomyłkę — czy można cofnąć?}

O: Usunięcie wydarzeń jest \textbf{nieodwracalne} przez interfejs graficzny.
W nagłym przypadku możesz przywrócić wydarzenie bezpośrednio z bazy danych
przez SSH (brak miękkiego kasowania). Skontaktuj się z dev teamem.

\bigskip
\textbf{P: Chcę dodać nowe miasto — jak to zrobię?}

O: Dodawanie nowych miast wymaga interwencji na poziomie bazy danych lub
konfiguracji backendu. Skontaktuj się z developerem aplikacji.

\bigskip
\textbf{P: Użytkownik prosi o usunięcie jego konta (żądanie RODO)?}

O: Usunięcie konta wymaga interwencji bezpośrednio w bazie danych. Skontaktuj
się z developerem. Przed usunięciem upewnij się, że użytkownik nie ma aktywnych
rejestracji ani otwartych płatności.

\bigskip
\textbf{P: Jak zmienić cenę wydarzenia po tym, jak użytkownicy się już
zapisali?}

O: Edytuj wydarzenie przez \texttt{/admin/edit-event/:id} i zmień cenę.
Istniejące rejestracje zachowują cenę z momentu zapisu. Nowe rejestracje
będą po nowej cenie.

\bigskip
\textbf{P: Jak wyeksportować listę uczestników danego wydarzenia?}

O: Eksport listy uczestników na poziomie interfejsu nie jest dostępny
w obecnej wersji. Możesz pobrać dane bezpośrednio z bazy danych przez SSH.
Skontaktuj się z developerem.

\bigskip
\textbf{P: Ile miejsca zajmują logi na serwerze?}

O: Sprawdź: \texttt{du -sh /opt/kenaz/logs/} po zalogowaniu przez SSH.
Jeśli katalog przekracza kilkaset MB, rozważ archiwizację starszych logów.

\bigskip
\textbf{P: Czy użytkownicy są informowani o weryfikacji lub odrzuceniu
płatności?}

O: Status płatności jest widoczny w aplikacji — użytkownik widzi aktualny
status na stronie panelu (\texttt{/panel}) i na stronie płatności
(\texttt{/manual-payment/:id}). System nie wysyła automatycznych emaili —
jeśli chcesz poinformować użytkownika, zrób to ręcznie.

% ══════════════════════════════════════════════════════════════════════════════
\clearpage
\section*{Słownik pojęć administratora}
\addcontentsline{toc}{section}{Słownik pojęć administratora}
\markboth{Słownik pojęć administratora}{}

\begin{longtable}{@{}L{4cm}L{11cm}@{}}
\toprule
\rowcolor{kenaznavy!8}
\textbf{Pojęcie} & \textbf{Definicja} \\
\midrule
\endhead
\bottomrule
\endfoot
\textbf{BANNED}            & Status konta zablokowanego przez administratora; użytkownik traci dostęp do płatnych funkcji \\
\rowcolor{tableshade}
\textbf{Kafelek admina}    & Rozszerzony panel informacyjny po kliknięciu użytkownika na liście \texttt{/admin/all-users}; zawiera dane o koncie, aktywności, finansach i oczekujących akcjach \\
\textbf{Korekta manualna}  & Ręczny wpis bilansu finansowego dodany przez administratora poza normalnym systemem transakcji \\
\rowcolor{tableshade}
\textbf{Log audytowy}      & Plik tekstowy zapisujący zdarzenia systemowe z datą, godziną, poziomem i treścią komunikatu \\
\textbf{Manualna płatność} & Przelew bankowy wykonany przez użytkownika, wymagający ręcznej weryfikacji przez administratora \\
\rowcolor{tableshade}
\textbf{PENDING}           & Status konta oczekującego na akceptację administratora \\
\textbf{Podgląd na żywo}   & Funkcja formularza tworzenia ikonek — prezentuje wygląd tworzonej ikony w czasie rzeczywistym \\
\rowcolor{tableshade}
\textbf{Promote}           & Operacja nadania użytkownikowi uprawnień administratora, dostępna pod adresem \texttt{/admin/promote} \\
\textbf{Rola}              & Poziom uprawnień: Guest (gość), Member (członek), Admin (pełny dostęp do panelu) \\
\rowcolor{tableshade}
\textbf{Standalone}        & Tryb działania PWA na iOS — aplikacja działa bez paska adresu Safari; istotne przy testowaniu OAuth \\
\textbf{Status konta}      & Stan konta: Active (aktywny), Pending (oczekujący na akceptację), Banned (zablokowany) \\
\rowcolor{tableshade}
\textbf{Weryfikacja płatności} & Proces ręcznego potwierdzenia przez administratora, że przelew bankowy dotarł i zgadza się z oczekiwaną kwotą \\
\end{longtable}

\vfill
\begin{center}
  {\color{kenaznavy!25}\rule{0.6\linewidth}{0.5pt}\par}
  \vspace{0.4cm}
  {\sffamily\small\color{kenazdim!70}
    Instrukcja Administratora — Kenaz Centrum, wersja 2.0, luty 2026.\par
    Dokument przeznaczony wyłącznie dla administratorów systemu.\par}
\end{center}

\end{document}
